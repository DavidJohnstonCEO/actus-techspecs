%%%%%%%%%%%%%%%%%%%%%%%%%%%%%%%%%%%%%%%%%%%%%%%%%%%%%%%%%%%%%%%%%%%%%%%%%%%%%%%%
%%%%%%%%%%%%                THIS IS THE                           %%%%%%%%%%%%%%
%%%%%%%%%%%%     ACTUS TECHNICAL SPECIFICATION DOCUMENT           %%%%%%%%%%%%%%
%%%%%%%%%%%%                                                      %%%%%%%%%%%%%%
%%%%%%%%%%%%              ----------------                        %%%%%%%%%%%%%%
%%%%%%%%%%%%        Copyright (C) 2016 - present by               %%%%%%%%%%%%%%
%%%%%%%%%%%%      ACTUS Financial Research Foundation             %%%%%%%%%%%%%%
%%%%%%%%%%%%             -----------------                        %%%%%%%%%%%%%%
%%%%%%%%%%%%      Please see distribution for license             %%%%%%%%%%%%%%
%%%%%%%%%%%%%%%%%%%%%%%%%%%%%%%%%%%%%%%%%%%%%%%%%%%%%%%%%%%%%%%%%%%%%%%%%%%%%%%%


%%%%%%%%%%%%%%%%%%%%%%% settings %%%%%%%%%%%%%%%%%%%%%%

% ---------------------- general ----------------------
\documentclass[9pt,oneside]{amsart}
\usepackage{multicol}
\usepackage[a4paper,
            width=170mm,
            top=18mm,
            bottom=22mm,
            includeheadfoot]{geometry}
\usepackage[bookmarks=true,
            unicode=true,
            pdftitle={ACTUS Technical Specification},
            pdfauthor={ACTUS Financial Research Foundation},
            pdfkeywords={ACTUS, Financial Contracts, Algorithmic Contracts, Technical Specification},
            pdfborder={0 0 0.5 [1 3]}]{hyperref}


% ---------------------- language ----------------------
\usepackage[english]{babel}


% ---------------------- floats ----------------------
\usepackage{graphicx}
\usepackage{float}
\usepackage{longtable}


% ---------------------- math ----------------------
\usepackage{amsmath}
\usepackage{amssymb}
\usepackage{amsthm}
\newtheorem{example}{Example}


% ---------------------- custom tables ----------------------
\newenvironment{states}[1]{
	\hfill % force subsection before longtable
	\begin{longtable}{| p{0.05\textwidth} | p{0.48\textwidth} |  p{0.43\textwidth} |}
	\multicolumn{3}{c}{\textbf{#1: State Variables Initialization}}\\
	\hline
	\textbf{State} & \textbf{Initialization per $t_0$} & \textbf{Comments} \\
	\hline
	\endfirsthead
	\multicolumn{3}{c}{\textit{Continued from previous page}} \\
	\hline
	\textbf{State} & \textbf{Initialization per $t_0$} & \textbf{Comments} \\
	\hline
	\endhead
	\hline \multicolumn{3}{r}{\textit{Continued on next page}} \\
	\endfoot
	\endlastfoot
}{%
	\hline
	\end{longtable}
}

\newenvironment{schedule}[1]{
	\hfill % force subsection before longtable
	\begin{longtable}{| p{0.05\textwidth} | p{0.5\textwidth} |  p{0.4\textwidth} |}
	\multicolumn{3}{c}{\textbf{#1: Contract Schedule}}\\
	\hline
	\textbf{Event} & \textbf{Schedule} & \textbf{Comments} \\
	\hline
	\endfirsthead
	\multicolumn{2}{c}{\textit{Continued from previous page}} \\
	\hline
	\textbf{Event} & \textbf{Schedule} & \textbf{Comments} \\
	\hline
	\endhead
	\hline \multicolumn{2}{r}{\textit{Continued on next page}} \\
	\endfoot
	\endlastfoot
}{%
	\hline
	\end{longtable}
}

\newenvironment{functions}[1]{
	\hfill % force subsection before longtable
    	\begin{longtable}{| p{0.05\textwidth} | p{0.42\textwidth} |  p{0.48\textwidth} |}
	\multicolumn{3}{c}{\textbf{#1: State Transition Functions and Payoff Functions}}\\
	\hline
	\textbf{Event} & \textbf{Payoff Function} & \textbf{State Transition Function}\\
	\hline
	\endfirsthead
	\multicolumn{2}{c}{\textit{Continued from previous page}} \\
	\hline
	\textbf{Event} & \textbf{Payoff Function} & \textbf{State Transition Function}\\
	\hline
	\endhead
	\hline \multicolumn{2}{r}{\textit{Continued on next page}} \\
	\endfoot
	\endlastfoot
}{%
	\hline
    	\end{longtable}
}


% ---------------------- custom notation ----------------------
\newcommand{\Real}{\mathbb{R}}
\newcommand{\Nat}{\mathbb{N}}
\newcommand{\svar}[2]{\textbf{#1}_{#2}}
\newcommand{\attr}[1]{\texttt{#1}}
\newcommand{\stf}[2]{STF\_#1\_#2()}
\newcommand{\pof}[2]{POF\_#1\_#2()}
\newcommand{\dfl}[1]{D(\textbf{Prf}_{#1})}
\newcommand{\sgn}{R(\attr{CNTRL})}
\newcommand{\sdl}[3]{S(#1,#2,#3)}
\newcommand{\vsdl}[3]{\vec{S}(#1,#2,#3)}
\newcommand{\yfr}[2]{Y(#1,#2)}
\newcommand{\yfrfunc}{Y}
\newcommand{\ann}[5]{A(#1,#2,#3,#4,#5)}
\newcommand{\obs}[3]{O^{#1}(#2,#3)}
\newcommand{\obsfull}[5]{O^{#1}(#2,#3,#4,#5)}
\newcommand{\obsfunc}[1]{O^{#1}}
\newcommand{\cldev}[3]{U^{ev}(#1,#2 \mid\{#3\})}
\newcommand{\cldsv}[4]{U^{sv}(#1,#2,\svar{#3}{} \mid\{#4\})}
\newcommand{\cldsvs}[3]{U^{sv}(#1,#2,\svar{#3}{})}
\newcommand{\cldca}[2]{U^{ca}(#1,#2)}
\newcommand{\cldfunc}[1]{U^{#1}}
\newcommand{\undef}{\varnothing}
\newcommand{\tmax}{t^{max}}
\newcommand{\tev}[1]{\tau(#1)}
\newcommand{\fev}[1]{f(#1)}
\newcommand{\payoff}[2]{F(#1,#2)}


% ---------------------- misc ----------------------
\usepackage{verbatim}
\usepackage{natbib}
\setlength\parindent{0pt}


% ---------------------- versioning ----------------------
\newcommand{\VersionNumber}{unknown revision}
\IfFileExists{build_options.tex}{\input{build_options.tex}}


%%%%%%%%%%%%%%%%%%%%%%% titlepage %%%%%%%%%%%%%%%%%%%%%%
\def\doctitle{ACTUS: The algorithmic representation of financial contracts}
\title{\doctitle \\
      {\smaller \textbf{Version \VersionNumber}}}

\author{
	Nils Bundi\\
	ACTUS Financial Research Foundation\\
	info@actusfrf.org
}


%%%%%%%%%%%%%%%%%%%%%%% headers and footers %%%%%%%%%%%%%%%%%%%%%%
\usepackage{fancyhdr}
\pagestyle{fancy}
\fancyhead{} % clear header
\fancyfoot{} % clear footer
\renewcommand{\headrulewidth}{0pt} % no header rule
\addtolength\footskip{3mm} % add space between main text and footer text
\lhead{\doctitle}
\rhead{\thepage}
\lfoot{Copyright \copyright\space 2018--present by ACTUS Financial Research Foundation}
\rfoot{\VersionNumber}


%%%%%%%%%%%%%%%%%%%%%%% front matter %%%%%%%%%%%%%%%%%%%%%%

\begin{document}

%\begin{abstract}
%
%\end{abstract}

\maketitle


% ---------------------- about ----------------------
\section*{About this document}\label{sec:about}

This document provides the technical specifications of the Algorithmic Contract Types Unified Standards (ACTUS). It is developed, maintained, and released by the ACTUS Financial Research Foundation and provided by the same to the ACTUS Users Association under the terms of the open source license with which the document is published from time to time.


% ---------------------- versions ----------------------
\section*{Versions}\label{sec:version}

This document is versioned according to the following pattern: [major].[minor]-[revision]-[date] where [major] and [minor] are integers marking major and minor release, [revision] indicates the current revision in form of the respective git commit hash (short form), and [date] gives the respective date of the revision. Releases are recorded in the following table.

\begin{table}[H]
  \centering
  \begin{tabular}{| p{0.1\textwidth} | p{0.1\textwidth} | p{0.75\textwidth} |}
  \hline
  Date & Version & Description \\
  \hline
  2018-11-01 & 1.0 & First version of the technical specification document containing specifications for the "initial" 18 contracts.\\
  \hline
  ??? & 1.1 & Streamlind with dictionary updates
	\begin{itemize}
		\item Fixed various naming conventions
		\item Aligned state variable names with dictionary
		\item Added \attr{ContractStructure} and "Composition"-section
		\item Added \attr{settlementCurrency} attribute and updated \attr{POF} accordingly
		\item Removed default convention and updated \attr{POF} accordingly
		\item Moved Taxonomy, Event, State, Contract Role definitions to dictionary
	\end{itemize} \par
		Fixed various bugs and inconsistencies. \\
  \hline
  \end{tabular}
\end{table}


% ---------------------- acknowledgments ----------------------
\section*{Acknowledgements}\label{sec:ack}

We would like to acknowledge all members of the \textit{ACTUS Users Association} who contribute a lot of their time and expertise to the development, review, and testing of the ACTUS standards, in general, and this document, in particular. Without their valuable contributions the ACTUS standards would not exist in the form as they currently do.


%%%%%%%%%%%%%%%%%%%%%%% table of contents %%%%%%%%%%%%%%%%%%%%%%

% ---------------------- new page ----------------------
\newpage
\tableofcontents



%%%%%%%%%%%%%%%%%%%% main body %%%%%%%%%%%%%%%%%%%%

% ---------------------- new page ----------------------
\newpage

% ---------------------- 2-columns ----------------------
\setlength{\columnsep}{20pt}
\begin{multicols}{2}


%%%%%%%%%%%%%%%%%%%% section: introduction %%%%%%%%%%%%%%%%%%%%

\section{Introduction}\label{sec:intro}

Financial contracts are legal agreements between two (or more) counterparties on the exchange of future cash flows. Such legal agreements are defined unambiguously by means of a set of contractual terms and logic. As a result, financial contracts can be described mathematically and represented digitally as machine readable algorithms.

The benefits of representing financial contracts digitally are manifold; Traditionally, transaction processing has been a field in which tremendous efficiency gains could be realized by the introduction of \textit{machines} and machine readable contracts. Or, financial analytics by nature of the domain builds on the availability of computable representations of these agreements where for reasons of tractability often times analytical approximations are used. Recently, the rise of distributed ledger and blockchain technologies and the various use cases for \textit{smart contracts} has opened up new possibilities for \textit{natively digital} financial contracts.

In general, the exchange of cash flows between counterparties follows certain patterns. A typical cash flow exchange pattern is a \textit{bullet loan} contract where principal is exchanged initially followed by cyclical interest payments and the principal is paid back (in a lump sum) at maturity of the contract. While the principal payments are fixed a variety of flavours exist for how the cyclical interest payments are determined and/or paid. As an example, interest payments may be due monthly, annually or according to arbitrary periods, they may be determined based on fixed or variable rates, different year fraction calculation methods may be used or there might be no interest due at all. Another popular pattern is that of \textit{amortizing loans} for which, as opposed to bullet loans, principal may be paid out and paid back in portions of fixed or variable amounts and according to cyclical or custom schedules. Other types of financial contracts include but are not limited to \textit{shares}, \textit{forwards}, \textit{options}, \textit{swaps}, \textit{credit enhancements}, \textit{repurchase agreements}, \textit{securitization}, etc. By focusing on the main distinguishing features, ACTUS describes the vast majority of all financial contracts with a set of about 32 generalized cash flow exchange patterns or Contract Types (CTs), respectively. 
The ACTUS taxonomy (\url{https://github.com/actusfrf/actus-dictionary/blob/master/actus-dictionary-taxonomy.json}) provides a classification system organizing financial contracts according to their distinguishing cash flow patterns. Apart from this classification system the taxonomy also includes a description of and real-world instruments covered for each contract.

On the other hand, the legal agreements in financial contracts represent purely deterministic logic or the \textit{mechanics of finance}, in other words. That is, a financial contract defines a fixed set of rules and conditions under which, given any external variables, the cash flow obligations can be determined unambiguously. For instance, in a \textit{fixed rate loan} the cash flow obligations are defined explicitly. At the same time, a \textit{variable rate loan} defines explicitly the rules under which the variable rate is fixed going forward such that the cash flow obligations can be derived unambiguously going forward. The same holds true for \textit{derivative contracts} where the cash flow obligations arise given some underlying \textit{reference instrument}. Similarly, for analytical purposes, given some assumption of the evolution of this reference instrument the cash flow obligations \textit{conditioned on} this assumption can be derived unabiguously.

The properties of financial contracts described above build the foundation for a standardized, deterministic algorithmic description of the cash flow obligations arising from such agreements. Thereby, this description is technology agnostic and supports all use cases necessary for this very standard to be used throughout all finance functions from front office to back office and covering pricing, deal origination, transaction processing, as well as analytics, in general, and liquidity projections, valuation, P\&L calculations and projections, and risk measurement and aggregation, in particular. Furthermore, this standard builds a formidable basis for distributed ledger-powered, natively digital \textit{financial state machines} or \textit{smart contracts}, in other words.

In this document, we provide the technical specification of the ACTUS standards or the mathematical description of financial contracts, in other words. We start by providing some basic notations used throughout the document followed by an introduction of the generic functions upon which financial contracts build. We continue in the following sections with an introduction of some additional foundational concepts \textit{Composition}, and \textit{Risk Factor Observer} and \textit{Child Contract Observer}. Finally, we define the various ACTUS contracts in the last section.


%%%%%%%%%%%%%%%%%%%% section: notations %%%%%%%%%%%%%%%%%%%%

\section{Notations}\label{sec:notations}

\subsection{Contract Attributes}\label{sec:attributes}

Contract Attributes (attributes) represent the legal contractual terms that define the exchange of cash-flows of a financial contract. These attributes are defined and described in the ACTUS dictionary (\url{https://github.com/actusfrf/actus-dictionary/blob/master/actus-dictionary-terms.json}). Throughout this document attributes are referenced by their short name according to the dictionary. Further, vector-type attributes may be indexed with a subscript indicating that a specific vector-element is referenced.

\begin{example}[Contract Attribute]
The ACTUS attribute \textit{Initial Exchange Date} is referenced in short form \attr{IED}.
\end{example}


\begin{example}[Element of Vector-Type Attribute]
The ACTUS attribute \textit{Array Cycle Anchor Date of Principal Redemption} is a vector-type attribute and referenced as \attr{ARPRANX}. The $i$-th element of the vector is represented by $\attr{ARPRANX}_i$.
\end{example}


\subsection{$\undef$-Operator}\label{sec:undef}

The $\undef$-operator is used to indicate that a certain property is undefined or, in other words, that no value has been assigned to the respective property. In particular, for optional contract attributes it means that the attribute is not defined and for schedule times (see section \ref{sec:schedule}) it means that the respective schedule is empty, i.e. no schedule time defined.

\begin{example}[Undefined Attribute]
$\attr{IPANX}=\undef$ indicates that attribute \attr{IPANX} is undefined.
\end{example}

\begin{example}[Empty Schedule]
$\vec{t}^{IP}=\undef$ means the same as $\vec{t}^{IP}=\{\}$, with $\{\}$ the empty set, and states that the IP schedule $\vec{t}^{IP}$ does not contain a schedule time.
\end{example}


\subsection{$t_0$-Time}\label{sec:t0time}

$t_0$ represents \attr{SD} of a contract and marks the time as per which the terms and implied state of a contract is represented. In general, from the contractual logic we are able to derive any contractual events and resulting states for any time $t>t_0$ but not for times $s<t_0$.


\subsection{State Variables}\label{sec:statevarsnotat}

State Variables (states) describe the state of a contract at a certain point in time $t$ during its lifetime. Examples of such states are the (outstanding) Notional Principal, the applicable Nominal Interest Rate, or the current Contract Performance. The ACTUS dictionary (\url{https://github.com/actusfrf/actus-dictionary/blob/master/actus-dictionary-states.json}) defines all states and provides further information on their data type, format, etc.

In general, states represent certain terms of a contract that change along the contract lifetime according to either scheduled events or unscheduled events. Therefore, states representing a contractual term carry the exact same names as their term-counterpart.

States are written in their short form representation with first letter capitalized, printed in bold, and indexed with time.

\begin{example}[State Variables]
$\svar{Nt}{t}$ refers to the state \textit{Notional Principal} observed at time $t$.
\end{example}


\subsection{Contract Events}\label{sec:events}

A Contract Event (event) $e_t^k$ refers to any contractually scheduled or unscheduled event at a certain time $t$ and of a certain type $k$. Contract events mark specific points in time during the lifetime of a contract at which a cash flow is being exchanged (see section \ref{sec:pof}) or the states of the contract are being updated (see section \ref{sec:stf}). The dictionary lists and describes all the event types $k$ defined by the ACTUS standards (\url{https://github.com/actusfrf/actus-dictionary/blob/master/actus-dictionary-event-types.json}). Throughout this document event types $k$ are written in the short form as defined in the dictionary.

As an event always has an associated event time $t$ and payoff $c\in\Real$ we define two operators allowing to retrieve these quantities for any single event $e_t^k$ or set of events $\{e_t^k,e_s^j, ...\}$ as follows;

{$\begin{aligned}
	\tev{x} &= \begin{cases} t & \text{if}\quad x=e_t^k \\
				\{t,s,...\} & \text{else if}\quad x=\{e_t^k,e_s^j,...\} \end{cases} \\
	\fev{x} &= \begin{cases} c & \text{if}\quad x=e_t^k \\
				\{c_1, c_2, ...\} & \text{else if}\quad x=\{e_t^k,e_s^j,...\} \end{cases}
\end{aligned}$}

with $c_1=\fev{e_t^k}, c_2=\fev{e_s^j}, ...$.

\begin{example}[Contract Events]
The \textit{Initial Exchange Date} event with event time $s$ is written as $e_s^{IED}$ with $\tev{e_s^{IED}}=s$ and $\fev{e_s^{IED}}=c$ where for any contract \attr{CT} $c=\pof{IED}{\attr{CT}}$.
\end{example}


\subsection{State Transition Functions}\label{sec:stf}

State Transition Functions (STF) define the transition of states from a pre-event to a post-event state when a certain event $e_t^{k}$ applies. Thereby, the pre-event and post-event times are indexed with $t^-$ and $t^+$, respectively. 

These functions are specific to a certain event and contract. STFs are written according to the following pattern \stf{[event type]}{[contract type]} where [event type] and [contract type] refer to the respective event type and contract to which the STF belongs.

\begin{example}[State Transition Functions]
The STF for an IP event and PAM contract is written as \stf{IP}{PAM} and maps e.g. state \textit{Accrued Interest} from a pre-event state $\svar{Ipac}{t^-}$ to post-event state $\svar{Ipac}{t^+}$.
\end{example}


\subsection{Payoff Functions}\label{sec:pof}

Payoff Functions (POF) define how the cash flow $c\in\Real$ for a certain event $e_t^k$ is being derived from current states and from the contract terms. If necessary, the resulting cash flow can be indexed with the event time $c_t$. These functions are specific to a certain event and contract. POFs are written according to the following pattern \pof{[event type]}{[contract type]} where [event type] and [contract type] refer to the respective event and contract to which the STF belongs.

\begin{example}[Payoff Functions]
The POF for an IP event $e_t^{IP}$ and PAM contract is written as \pof{IP}{PAM} with $\fev{e_t^{IP}}=\pof{IP}{PAM}$.
\end{example}


\subsection{Date/Time}\label{sec:time}

ACTUS builds on the ISO 8601 date/time format. Hence, dates are generally expressed in the following format: [YYYY]-[MM]-[DD]T[hh]:[mm]:[ss].

Time zone information is currently not supported.

A special case is \textit{midnight}. ISO 8601 recognizes both times 00:00:00 and 24:00:00 each referring to midnight. Yet, while 24:00:00 refers to the end of one day, 00:00:00 refers to the beginning of the following day. In ACTUS the interpretation is the same why the time period (measured in any time unit) between the two points in time will always be zero.

For brevity, we use the term \textit{time} for a specific date-time variable.

\textbf{A note on implementation:} As many implementations of the ISO 8601 format do not support the 24:00:00 format we interpret the timestamp \verb'23:59:59' as midnight.


\subsection{Event Sequence}\label{sec:eventseq}

Contract Events of different types may occur at the same time, i.e. exactly the same point in time. In this case, the sequence of evaluating their STF and POF is decisive for the resulting cash flows and state transitions. Hence, we use an event sequence indicator that can be found for each event in the event-dictionary and implies the order of executing different events at the exact same time.


\subsection{Contract Lifetime}\label{sec:lifetime}

The lifetime of an ACTUS contract is the time period of its existence from the perspective of the analyzing user. For every point in time during its lifetime, an ACTUS contract can be analyzed in terms of current state and future cash flows.\\

The lifetime of a contract starts with its \attr{SD} and ends with $\min(MD, AMD, PR^*, STD, TD,\tmax)$.\\

Note that $PR^*$ refers to the PR event of a maturity contract after which \textbf{Nt}=0.0 (i.e. at which the remaining outstanding principal is redeemed). Further, \attr{MD}, \attr{AMD}, and PR(\textbf{Nt}=0.0) in the definition above do only apply for maturity contracts but have to be considered infinity in all other cases. Similarly, \attr{STD} only applies for certain contracts and is considered infinity for all others. Finally, $\tmax$ is a parameter that may be used to restrict the considered lifetime in an analysis. In particular, this parameter is used for contracts that do not have a \textit{natural} end to their lifetime such as STK.


%%%%%%%%%%%%%%%%%%%% section: utility functions %%%%%%%%%%%%%%%%%%%%

\section{Utility Functions}\label{sec:utils}


\subsection{Schedule}\label{sec:schedule}

A schedule is a function $S$ mapping times $s,T$ with $s<T$ and cycle $c$ onto a sequence $\vec{t}$ of cyclic times

\[
	\sdl{s}{c}{T}=\vec{t}=\begin{cases} \{\} & \text{if}\quad s=\undef\land T=\undef\\
					s & \text{else if}\quad T=\undef\\
					(s,T) & \text{else if}\quad c=\undef\\
					(s=t_1,...,t_n=T) & \text{else} \end{cases}
\]

with $t_i<t_{i+1}, i=1,2,...$. While the schedule function can be used to create arbitrary sequences of times, it is usually used to generate sequences of cyclic events $\vec{t}^k$ of a certain type $k$, e.g. $k=IP$ for interest payment events (cf. table \ref{tbl:events}) and the following build inputs to the function

\begin{itemize}
	\item[$s$] $=k\attr{ANX}$ with $k\attr{ANX}$ attribute cycle anchor date of event type $k$

	\item[$c$] $=k\attr{CL}$ with $k\attr{CL}$ event type $k$'s schedule cycle

	\item[$T$] $=\attr{MD}$ with \attr{MD} the contract's maturity
\end{itemize}

Thereby, cycles $k\attr{CL}$ have format $NPS$ where

\begin{itemize}
	\item[$N$] is an integer
	\item[$P$] is a time period unit (D=Day, W=Week, M=Month, Q=Quarter, H=Half Year, Y=Year)
	\item[$S$] is a \textit{stub} information ($+$=long last stub, $-$=short last stub)
\end{itemize}

Further, the last stub is defined as follows

\begin{itemize}
	\item[if] $t_{n-1}+c=T \lor S=$'-' then no stub correction applies

	\item[else] $t_n$ is removed from the schedule
\end{itemize}

The sequence of schedule times $\vec{t}^k$ may also be influenced by the \attr{EOF} and \attr{BDC} conventions and the full function syntax becomes $\sdl{s}{c}{T, \attr{EOMC}, \attr{BDC}}$. Due to such effects the sequence of schedule times can be non-equidistant or, in other words, $t_i^k-t_{i-1}^k\neq t_j^k-t_{j-1}^k, i\neq j$.\\

Note that for brevity we will omit the \attr{EOMC} and \attr{BDC} function arguments throughout this document.


\subsection{Array Schedule}\label{sec:arrayschedule}

Array Schedules are defined by vector-valued inputs $\vec{s}=(s_0,s_1,...,s_m)$ and $\vec{c}=(c_0,c_1,...,c_m)$ to the array schedule function

\begin{multline*}
	\vsdl{\vec{s}}{\vec{c}}{T} = (\sdl{s_0}{c_0}{s_1-c_0},\\
					\sdl{s_1}{c_1}{s_2-c_1},...,\sdl{s_m}{c_m}{T})
\end{multline*}


Hence, array schedules are a generalization for regular schedules which coincide for $m=1$. In accordance with regular schedules \attr{EOMC} and \attr{BDC} conventions also apply here.


\subsection{End Of Month Shift Convention}\label{sec:eomc}

For schedules $\vec{t}^k$ starting at time $s$ which marks the end of a month with 30 or less days, e.g. April 30, and with a cycle $c$ being a multiple of 1M- attribute \attr{EOM} defines whether the schedule times are to fall on the 30th of all months (same day) or the 31st (end of month).\\

More specifically, \attr{EOM} has an effect on a schedule $\vec{t}^k$ only if:

\begin{itemize}
	\item[$s$] is the last day of a month with less than 31 days (Feb, April etc.)

	\item[$c$] $=NPS$ with $P\in(M, Q, H or Y)$
\end{itemize}

As per the DD \attr{EOM} can take one of the following values:
\begin{itemize}
	\item[EOM] (EndOfMonth): times $t_i,i=1,2,...,n-1$ are moved to the end of the respective months

	\item[SD] (SameDay): times $t_i,i=1,2,...,n-1$ remain unchanged except in February, where it will go to the last day if the day of month of time $s$ is higher than the number of days of February
\end{itemize}


\subsection{Business Day Shift Convention}\label{sec:bdc}

In general, contract events are scheduled for business days only. Therefore, the \attr{BDC} convention defines how scheduled times $t_i,i=1,2,...,n-1$ are shifted in case they fall on a non-business day:

\begin{itemize}
	\item[NULL:] No shift

	\item[SCF:] Shift/Calculate following: The event is shifted to the following non working day. Calculation of the event happens after the shift

	\item[SCMF:] Shift/Calculate modified following: The event is shifted to the following non working day. However, if the following day happens to fall into the next month, then take preceeding non-working day. Calculation of the event happens after the shift

	\item[CSF:] Calculate/Shift following: Same like SCF however calculation of the event happens before the shift

	\item[CSMF:] Calculate/Shift modified following: Same like SCMF however calculation of the event happens before the shift

	\item[SCP:] Shift/Calculate preceding: The event is shifted to the last preceding non working day. Calculation of the event happens after the shift

	\item[SCMP:] Shift/Calculate modified preceding: The event is shifted to the last preceding non working day. However, if the preceding day happens to fall into the previous month, then take next non-working day. Calculation of the event happens after the shift

	\item[CSP:] Calculate/Shift preceding: Same like SCP however calculation of the event happens before the shift

	\item[CSMP:] Calculate/Shift modified preceding: Same like SCMP however calculation of the event happens before the shift
\end{itemize}


\subsection{Business Day Calendar}\label{sec:bdcal}

Whether a specific day is a business day (cf. previous section) is defined by attribute \attr{CLDR}. Such conventions generally depend on regional official holiday calendars. The Business Day Function interface allows determining for some \attr{CLDR} whether any time $t$ is a business day or not

\[
	B: t \mapsto \{true, false\}
\]

where $true$ indicates that $t$ is a business day and $false$ that it is a holiday.


\begin{example}
Two standard \attr{CLDR} implementations are the following
\begin{itemize}
	\item NoHoliday (default): every calendar day is a business day

	\item MondayToFriday: all weekdays Monday, Tuesday, Wednesday, Thursday, and Friday are business days
\end{itemize}
\end{example}


\subsection{Year Fraction Convention}\label{sec:yearfrac}

Interest income and other calculations are based on \textit{per annum} interest rates. Therefore, the year-fraction function interface $Y$ is used to calculate the \textit{fraction of a year} between any two times $s$ and $t$ with $t>s$ for which e.g. an (per annum) interest rate applies according to some day count convention \attr{DCC}

\[
	\yfrfunc: s,t,\attr{DCC} \mapsto \Real
\]

Note, the year fraction function interface only defines the structure of year fraction functions but not an actual implementation thereof, or the respective \attr{DCC}, respectively. Therefore, any \attr{DCC} can be implemented according to the interface above supporting user-defined year fraction functions.\\

For brevity we will omit the \attr{DCC} function argument wherever this does not lead to confusion.


\subsection{Contract Role Sign Convention}\label{sec:cntrl}

The two parties to a contract are defined through attributes \attr{CRID} and \attr{CPID}. The first is the party initially \textit{creating} the contract and the second is the counterparty, respectively. Thereby, both \attr{CRID}/\attr{CPID} can take any \textit{role} in the contract or, more specifically, they can be the lender or borrower in a loan (PAM), fixed receiver or payer in an interest rate swap (SWAPS), etc.\\

The \textit{role} of the \attr{CRID} is defined through attribute \attr{CNTRL}. The \textit{role} of \attr{CPID} is derived as the \textit{opposite} side to the contract. Apart from \attr{CNTRL} the attributes are \textit{neutral} to the \textit{role} of \attr{CRID} (or \attr{CPID}).\\

On the other hand, contractual cash flows generated by the POFs and certain states are \textit{role-sensitive}. That is, from the perspective of the \attr{CRID} these quantities represent either claims or obligations. Contract Role Sign function $R$ maps the \attr{CNTRL} attribute into $+1$ indicating a claim or $-1$ indicating an obligation

\[
	R : \attr{CNTRL} \rightarrow \{-1, +1 \}
\]

When multiplying with a cash flow $x$ the Contract Role Sign function thereby defines the direction of that flow:

\begin{itemize}
	\item[$x>0$:] $x$ flows from \attr{CPID} to \attr{CRID}

	\item[$x<0$:] $x$ flows from \attr{CRID} to \attr{CPID}
\end{itemize}

Table \ref{tbl:cntrl} defines the domain of the Contract Role Sign function, i.e. the range of attribute \attr{CNTRL}, with meaning and Contract Role Sign to which the function maps.


% ---------------------- table: contract roles ----------------------
\begin{table}[H]
	\centering
	\begin{tabular}{| p{0.5in}p{1.5in}p{0.2in} |}
	\hline
	\textbf{Value} & \textbf{Meaning} & $\textbf{R}$ \\
	\hline
	RPA & Real position asset & +1 \\
	\hline
	RPL & Real position liability & -1 \\
	\hline
	LG & Long position & +1 \\
	\hline
	ST & Short position & -1 \\
	\hline
	BUY & Protection buyer & +1 \\
	\hline
	SEL & Protection seller & -1 \\
	\hline
	RFL & Receive first (or fixed) leg & +1 \\
	\hline
	PFL & Pay first (or fixed) leg & -1 \\
	\hline
	COL & Collateral instrument & +1 \\
	\hline
	GUA & The guarantor in a Guarantee & -1 \\
	\hline
	OBL & The obligee in a Guarantee & +1 \\
	\hline
	\end{tabular}
	\caption{Contract Role definitions.}
	\label{tbl:cntrl}
\end{table}


\subsection{Annuity Amount Function}\label{sec:annamount}

In an \textit{Annuity} contract (ANN) the annuity amount is paid regularly from the \textit{borrower} to the \textit{lender}. Thereby, the annuity amount is comprised of a principal repayment portion and an interest portion and and dimensioned such that the total nominal amount $n$ at time $t$ is fully repaid at maturity $T$ of the annuity. The Annuity Amount function $\ann$ computes the annuity amount as follows

\[
	\ann{s}{T}{n}{a}{r}=(n+a)\frac{\prod_{i=1}^{m-1}1+r\yfr{t_i,t_{i+1}}}{1+\sum_{i=1}^{m-1}\prod_{j=i}^{m-1}1+r\yfr{t_j}{t_{j+1}}}
\]

with $a$ the accrued interest as per time $s$, $r$ the actual interest rate, $t_i, i=1,2,...,m$ the schedule times $\inf t, t\in\vec{t}^{PR}\land t>s$, $m$ the number of times $t_i$, and $\vec{t}^{PR}$ the PR-event schedule times of the Annuity contract as described in section \ref{sec:ann}.


\subsection{Canonical Contract Payoff Function}

The canonical payoff of a contract $x$ is defined as the sum of all future event payoffs evaluated under current risk factor conditions, or

\[
  \payoff{x}{t} = \sum_{c\in C} c
\]

with $C=\fev{\cldev{x}{t}{\obs{rf}{\attr{i}}{s}=\obs{rf}{\attr{i}}{t}\forall i \wedge s>t}}$.


%%%%%%%%%%%%%%%%%%%% section: contract composition %%%%%%%%%%%%%%%%%%%%

\section{Contract Composition}\label{sec:composition}

The payoff of \textit{Combined Contracts}, see the taxonomy, is derived from certain quantities of child contracts (also called \textit{underlying instruments} or simply \textit{underlyers}). In general, such child contracts can be any ACTUS contract - Basic or Combined - as well as any number of contracts - a single contract or a set of contracts. Indeed, in reality this is what Option, Swap, Swaption, but also any kind of Asset/Mortgage/etc. backed securities represent; \textit{a hierarchical composition of different contracts linked by means of functional reationships}. This compositional approach provides maximum flexibility and, hence, allows capturing any real world use case. We here refer to a \textit{referenced} (i.e. of lower hierarchical level) contract as a \textit{child contract} and to a referencing (i.e. of higher hierarchical level) contract as a \textit{parent contract}.\\

The ACTUS dictionary defines attribute $\attr{CTST}$ which captures the child contract(s) as part of the parent contract's set of attributes. Thereby, attribute $\attr{CTST}$ is of type \textit{ContractStructure} which is a simple structure of the form (in JSON notation)

\begin{verbatim}
{
"ContractReference":[ r1, r2, ... ]
}
\end{verbatim}

where the \verb'r1', \verb'r2', ... refer to objects of type \textit{ContractReference} (in JSON notation)

\begin{verbatim}
{
"Object": ,
"Type": ,
"Role":
}
\end{verbatim}

with fields

\begin{itemize}
	\item[] \verb'Object': an object representing the reference
	\item[] \verb'Type': one of the values listed in table \ref{tbl:struct-types}
	\item[] \verb'Role': one of the values listed in table \ref{tbl:struct-roles}
\end{itemize}


% ---------------------- table: contract structure types ----------------------
\begin{table}[H]
	\centering
	\begin{tabular}{| p{1.2in}p{1.7in} |}
	\hline
	\textbf{Type} & \textbf{Explanation} \\
	\hline
	Contract & The reference represents an actual contract \\
	\hline
	ContractIdentifier & The reference represents an identifier of an actual contract \\
	\hline
	MarketObjectIdentifier & The reference represents the identifier of a market object (e.g. price feed) \\
	\hline
	LegalEntityIdentifier & The reference represents the identifier of a legal entity \\
	\hline
	ContractStructure & The reference represents a ContractStructure \\
	\hline
	\end{tabular}
	\caption{ContractReference types}
	\label{tbl:struct-types}
\end{table}


% ---------------------- table: contract structure roles ----------------------
\begin{table}[H]
	\centering
	\begin{tabular}{| p{1.0in}p{1.4in}p{0.5in} |}
	\hline
	\textbf{Role} & \textbf{Explanation} & \textbf{Contract} \\
	\hline
	Underlying & The reference represents a simple underlyer contract & FUTUR, OPTNS \\
	\hline
	FirstLeg & The reference represents the first leg contract & SWAPS \\
	\hline
	SecondLeg & The reference represents the second leg contract & SWAPS \\
	\hline
	CoveredContract & The reference represents a contract that is covered under the parent contract & CEG, CEC \\
	\hline
	CoveringContract & The reference represents a contract that covers for covering contracts under the parent contract & CEC \\
	\hline
	\end{tabular}
	\caption{ContractReference roles}
	\label{tbl:struct-roles}
\end{table}


We will use the following notation to query reference objects from the $\attr{CTST}$ attribute

\[
	\attr{CTST}_{role}^{type}(i)
\]

where $type$ identifies the references (\verb'r1', \verb'r2', ...) with \verb'type'=$type$, $role$ the reference with \verb'Role'=$role$, and $i$ identifies the $i$'th element of the queried references. For brevity, we will omit the index parameter $i$ which indicates that we address the first (and usually only) reference object queried.

\begin{example}[Underlying MarketObject-reference] The MarketObject reference of a simple Underlying e.g. to an Option contract is referenced as $\attr{CTST}_{Underlying}^{MarketObjectIdentifier}(1)$ or, in short form, as $\attr{CTST}_{Underlying}^{MarketObjectIdentifier}$.
\end{example}

\begin{example}[FirstLeg Contract-reference] The Contract object representing the first leg e.g. to a Swaps contract is referenced as $\attr{CTST}_{FirstLeg}^{Contract}(1)$ or, in short form, as $\attr{CTST}_{FirstLeg}^{Contract}$.
\end{example}


%%%%%%%%%%%%%%%%%%%% section: risk factor observer %%%%%%%%%%%%%%%%%%%%

\section{Risk Factor Observer}\label{sec:rfobserver}

The payoff of financial contracts always depends on the context in which it is evaluated and which is comprised of the following dimensions; counterparties, markets, and behavioral factors. We refer to these as the \textit{risk factors} to which financial contracts are exposed to. This indicates that these factors are source of uncertainty because financial contracts only reference the factors but their dynamics is outside the control of any contractual agreement. Thus, such factors have to be \textit{observed} and their changing states accounted for when evaluating the payoff of financial contracts. Therefore, we consider a standardized interface $\obsfull{o}{i}{t}{S}{M}$ that allows for \textit{observing}; (1) the state of a certain risk factor $i$ at any time $t$ if $o=$'rf'

\[
	\obsfunc{rf}: i,t,S,M \mapsto \Real
\]

and (2) contractual but non-scheduled events if $o=$'ev'

\[
	\obsfunc{ev}: i,k,t,S,M \mapsto \{e_t^{k},e_s^{k},...\}
\]

The parameters to the Risk Factor Observer interface are as follows:

\begin{itemize}
	\item[$i$]: the identifier of the risk factor observed

	\item[$k$]: the type of events observed

	\item[$t$]: the time (post) which to observe the risk factor

	\item [$S$]: the inner states of the contract at time $t$

	\item [$M$]: the contract terms of the contract as per time $t$
\end{itemize}

Note that the observer interface only defines the structure of an actual observer function but not the actual implementation. Thus, the interface allows for user-defined implementations of observer functions allowing e.g. for representing arbitrary assumptions on the evolution of future risk factor states which is key for any type of forward-looking analysis.

\begin{example}['rf'-Observer] The market-driven 3-month USD-Libor reference rate used as the variable rate in a variable rate loan contract is observed at any time $t$ through $\obs{rf}{\attr{MarketObjectCodeRateReset}}{t}$.
\end{example}

\begin{example}['rf'-Observer] Unscheduled (pre-) repayments of outstanding notional in a mortgage contract is observed at any time $t$ through $\obs{ev}{\attr{CID},\attr{PR}}{t}$.
\end{example}

For brevity we will omit the $S$ and $M$ function arguments wherever this does not lead to confusion.


%%%%%%%%%%%%%%%%%%%% section: child observer %%%%%%%%%%%%%%%%%%%%

\section{Child Contract Observer}\label{sec:cldobserver}

In order to evaluate the derived payoff of combined contracts, we consider a standardized interface $\cldfunc{o}$ that allows for \textit{observing} on the parent level; (1) all future events, w.r.t. time $t$, if $o=$'ev'

\[
	\cldfunc{ev}: i,t,a \mapsto \{e_v^{k},e_w^{l},...\}
\]

with $v,w>t$ and event types $k,l$ according to the schedule of the child contract, (2) a certain state variable $x$ if $o=$'sv'

\[
	\cldfunc{sv}: i,t,x,a \mapsto \Real,
\]

or (3) a particular contract attribute $x$ of the child contract if $o=$'ca'

\[
	\cldfunc{ca}: i,x \mapsto y
\]

with $y$ a variable of value type of the respective attribute as per DD.\\

The parameters to the Child Contract Observer interface are as follows:

\begin{itemize}
	\item[$i$]: the identifier of the child contract \textit{observed}

	\item[$t$]: for $o\in\{\textit{ev,sv}\}$ the time for which the respective quantity should be evaluated

	\item [$x$]: for $o\in\{\textit{sv,ca}\}$ the quantity to be evaluated

	\item [$a$]: for $o\in\{\textit{ev,sv}\}$ a set of contract attributes to which the evaluated quantity should be conditioned
\end{itemize}


Note that the observer interface only defines the structure of an actual observer function but not the actual implementation. Thus, the interface allows for user-defined implementations of observer functions allowing e.g. for using arbitrary data structures.

\begin{example}['ev'-Observer] The future events, w.r.t. time $t$, of the \textit{first leg} (i.e. child contract with \verb'Role'=$\texttt{FirstLeg}$) of a SWAPS contract with $\attr{CNTRL}=PFL$ (i.e. \textit{pay first leg}) can be evaluated as $\cldev{\attr{CTST}_{FirstLeg}^{Contract}}{t}{\attr{CNTRL}=RPL}$.
\end{example}

\begin{example}['sv'-Observer] The current state, w.r.t. time $t$, of state variable $\svar{Nt}{}$ of the \textit{first leg} (i.e. child contract with \verb'Role'=$\texttt{FirstLeg}$) of a SWAPS contract with \attr{CNTRL}=RFL (i.e. \textit{receive first leg}) can be evaluated as $\cldsv{\attr{CTST}_{FirstLeg}^{Contract}}{t}{Nt}{\attr{CNTRL}=RPA}$.
\end{example}

\begin{example}['ca'-Observer] The contract attribute \attr{MOC} of the child contract \textit{Child} (i.e. child contract with \verb'Role'=$\texttt{Underlying}$) of an OPTNS contract can be evaluated as $\cldca{\attr{CTST}_{Underlying}^{Contract}}{\attr{MOC}}$.
\end{example}

For brevity we will omit the $x$ and $a$ function arguments wherever this does not lead to confusion.



%%%%%%%%%%%%%%%%%%%% section: contract types %%%%%%%%%%%%%%%%%%%%

% ---------------------- 1-column ----------------------
\end{multicols}

% ---------------------- newpage ----------------------
\newpage

\section{Contract Types}\label{sec:contracts}



%%%%%%%%%%%%%%%%%%%% subsection: pam %%%%%%%%%%%%%%%%%%%%

\subsection{PAM: Principal At Maturity}\label{sec:pam}


% ---------------------- table: pam schedule ----------------------
\begin{schedule}{PAM}
	AD & $\vec{t}^{AD} = \left(t_0,t_1,...,t_n\right)$ & With $t_i,i=1,2,...$ a custom input \\
	\hline
	IED & $t^{IED} = \attr{IED}$ & \\
	\hline
	PR & $t^{PR} = \svar{Tmd}{t_0}$ & \\
	\hline
	PP & $\vec{t}^{PP} = \begin{cases} \undef & \text{if} \quad \attr{PPEF}=\text{'N'} \\
					(\vec{u},\vec{v}) & \text{else} \end{cases}$
		\par where \par
		{$\begin{aligned} \vec{u} &= \sdl{s}{\attr{OPCL}}{T^{MD}} \\
				\vec{v} &= \obs{rf}{\attr{PPMO}}{t} \end{aligned}$}
		 & with\par $s = \begin{cases} \undef & \text{if} \quad \attr{OPANX}=\undef \land \attr{OPCL}=\undef\\
					   \attr{IED}+\attr{OPCL} & \text{else if} \quad \attr{OPANX} = \undef \\
					   \attr{OPANX} & \text{else} \end{cases}$ \\
	\hline
	PY & $\vec{t}^{PY} = \begin{cases} \undef & \text{if} \quad \attr{PYTP}=\text{'O'} \\
						\vec{t}^{PP} & \text{else} \end{cases}$ & \\
	\hline
	FP & $\vec{t}^{FP} = \begin{cases} \undef & \text{if} \quad \attr{FER}=\undef \lor \attr{FER}=0 \\
					\sdl{s}{\attr{FPCL}}{T^{MD}} & \text{else} \end{cases}$
		& with\par $s = \begin{cases} \undef & \text{if} \quad \attr{FPANX}=\undef \land \attr{FPCL}=\undef\\
					   \attr{IED}+\attr{FPCL} & \text{else if} \quad \attr{FPANX} = \undef \\
					   \attr{FPANX} & \text{else} \end{cases}$ \\
	\hline
	PRD & $t^{PRD}= \attr{PRD}$  &  \\
	\hline
	TD & $t^{TD}= \attr{TD}$  &  \\
	\hline
	IP & $\vec{t}^{IP} = \begin{cases} \undef & \text{if} \quad \attr{IPNR}=\text{'O'} \\
						\sdl{s}{\attr{IPCL}}{T^{MD}} & \text{else} \end{cases}$
		& with\par $s = \begin{cases} \undef & \text{if} \quad \attr{IPANX}=\undef \land \attr{IPCL}=\undef\\
					\attr{IPCED} & \text{else if}\quad \attr{IPCED}\neq\undef \\
					   \attr{IED}+\attr{IPCL} & \text{else if} \quad \attr{IPANX} = \undef \\
					   \attr{IPANX} & \text{else} \end{cases}$ \\
	\hline
	IPCI & $\vec{t}^{IPCI} = \begin{cases} \undef & \text{if} \quad \attr{IPCED}=\undef \\
						\sdl{s}{\attr{IPCL}}{\attr{IPCED}} & \text{else} \end{cases}$
		& with\par $s = \begin{cases} \undef & \text{if} \quad \attr{IPANX}=\undef \land \attr{IPCL}=\undef\\
					   \attr{IED}+\attr{IPCL} & \text{else if} \quad \attr{IPANX} = \undef \\
					   \attr{IPANX} & \text{else} \end{cases}$ \\
	\hline
	RR & $\vec{t}^{RR} = \begin{cases} \undef & \text{if} \quad \attr{RRANX}=\undef \land \attr{RRCL}=\undef \\
					\vec{t} \setminus t^{RRY} & \text{else if} \attr{RRNXT} \neq \undef \\
					\vec{t} & \text{else} \end{cases}$ \par
		where $\vec{t}=\sdl{s}{\attr{RRCL}}{T^{MD}}$
		& with\par {$\begin{aligned} s &= \begin{cases} \attr{IED}+\attr{RRCL} & \text{if} \quad \attr{RRANX} = \undef \\
					   \attr{RRANX} & \text{else} \end{cases} \\
					     t^{RRY} &= \inf t \in \vec{t}\mid t>\attr{SD} \end{aligned}$} \\
	\hline
	RRF & $t^{RRF} = \begin{cases} \undef & \text{if} \quad \attr{RRANX}=\undef \land \attr{RRCL}=\undef \\
					\inf t \in \vec{t}\mid t>\attr{SD} & \text{else} \end{cases}$ \par
		where $\vec{t}=\sdl{s}{\attr{RRCL}}{T^{MD}}$
		& with\par $s = \begin{cases} \attr{IED}+\attr{RRCL} & \text{if} \quad \attr{RRANX} = \undef \\
					   \attr{RRANX} & \text{else} \end{cases}$ \\
  	\hline
	SC & $\vec{t}^{SC} = \begin{cases} \undef & \text{if} \quad \attr{SCEF}=\text{'000'} \\
					\sdl{s}{\attr{SCCL}}{T^{MD}} & \text{else} \end{cases}$
		& with\par $s = \begin{cases} \undef & \text{if} \quad \attr{SCANX}=\undef \land \attr{SCCL}=\undef\\
					   \attr{IED}+\attr{SCCL} & \text{else if} \quad \attr{SCANX} = \undef \\
					   \attr{SCANX} & \text{else} \end{cases}$ \\
	\hline
	CE & $\vec{t}^{CE} = t(e^{k}) | \svar{Prf}{t^-} \neq \svar{Prf}{t^+}, \forall k$  & \\
\end{schedule}


% ---------------------- table: pam states ----------------------
\begin{states}{PAM}
	$\svar{Tmd}{}$ & $\svar{Tmd}{t_0} = \attr{MD}$ & \\
	\hline
  	$\svar{Nt}{}$ & $\svar{Nt}{t_0} = \begin{cases} 0.0 & \text{if} \quad \attr{IED} > t_0 \\
							\sgn\times\attr{NT} & \text{else} \end{cases}$ & \\
	\hline
	$\svar{Ipnr}{}$ & $\svar{Ipnr}{t_0} = \begin{cases} 0.0 & \text{if} \quad \attr{IED} > t_0 \\
							\attr{IPNR} & \text{else} \end{cases}$ & \\
  	\hline
  	$\svar{Ipac}{}$ & $\svar{Ipac}{t_0} = \begin{cases} 0.0 & \text{if} \quad \attr{IPNR}=\undef \\
							\attr{IPAC} & \text{else if} \quad \attr{IPAC} \neq \undef \\
							\yfr{t^-}{t_0}\times\svar{Nt}{t_0}\times\svar{Ipnr}{t_0} & \text{else} \end{cases}$ &
			with $t^- = \sup t \in \vec{t}^{IP}\mid t<t_0$ \\
	\hline
  	$\svar{Fac}{}$ & $\svar{Fac}{t_0} = \begin{cases} 0.0 & \text{if} \quad \attr{FER}=\undef \\
					\attr{FEAC} & \text{else if} \quad \attr{FEAC} \neq \undef \\
					\yfr{t^-}{t_0}\times\svar{Nt}{t_0}\times\attr{FER} & \text{else if} \quad \attr{FEB}=\text{'N'} \\
					\frac{\yfr{t^{FP-}}{t_0}}{\yfr{t^{FP-}}{t^{FP+}}}\times\attr{FER} & \text{else} \end{cases}$ &
			with {$\begin{aligned} t^{FP-} &= \sup t \in \vec{t}^{FP}\mid t<t_0 \\
						t^{FP+} &= \inf t \in \vec{t}^{FP}\mid t>t_0 \end{aligned}$} \\
  	\hline
  	$\svar{Nsc}{}$ & $\svar{Nsc}{t_0} = \begin{cases} \attr{SCIXSD} & \text{if} \quad \attr{SCEF}=\text{'[x]N[x]'} \\
					1.0 & \text{else} \end{cases}$ & \\
  	\hline
  	$\svar{Isc}{}$ & $\svar{Isc}{t_0} = \begin{cases} \attr{SCIXSD} & \text{if} \quad \attr{SCEF}=\text{'I[x][x]'} \\
					1.0 & \text{else} \end{cases}$ & \\
	\hline
  	$\svar{Prf}{}$ & $\svar{Prf}{t_0} = \attr{PRF}$ &  \\
	\hline
	$\svar{Sd}{}$ & $\svar{Sd}{t_0} = t_0$ & \\
\end{states}


% ---------------------- table: pam functions ----------------------
\begin{functions}{PAM}
	AD & 0.0 & {$\begin{aligned}
				\svar{Ipac}{t^+} &= \svar{Ipac}{t^-} + \yfr{\svar{Sd}{t^-1}}{t}\svar{Ipnr}{t^-}\svar{Nt}{t^-}\\
				\svar{Sd}{t^+} &= t
			\end{aligned}$} \\
	\hline
	IED & $\obs{rf}{\attr{CURS}}{t}\sgn(-1)(\attr{NT}+\attr{PDIED})$
		& {$\begin{aligned}
			\svar{Nt}{t^+} &=\sgn\attr{NT} \\
			\svar{Ipnr}{t^+} &= \begin{cases} 0.0 & \text{if} \quad \attr{IPNR}=\undef \\
							\attr{IPNR} & \text{else} \end{cases} \\
			\svar{Ipac}{t^+} &= \begin{cases} \attr{IPAC} & \text{if} \quad \attr{IPAC} \neq \undef \\
							y\svar{Nt}{t^+}\svar{Ipnr}{t^+} & \text{if} \quad \attr{IPANX} \neq \undef \land \attr{IPANX}<t \\
							0.0 & \text{else} \end{cases} \\
			\svar{Sd}{t^+} &= t \end{aligned}$}\par
		with\par
		$y=\yfr{\attr{IPANX}}{t}$ \\
	\hline
	PR & $\obs{rf}{\attr{CURS}}{t}\svar{Nsc}{t^-}\svar{Nt}{t^-}$
		& {$\begin{aligned}
			\svar{Nt}{t^+} &= 0.0 \\
			\svar{Ipnr}{t^+} &= 0.0 \\
			\svar{Sd}{t^+} &= t \end{aligned}$} \\
	\hline
	PP & $\obs{rf}{\attr{CURS}}{t}\obs{rf}{\attr{OPMO}}{t}$
		& {$\begin{aligned}
			\svar{Ipac}{t^+} &= \svar{Ipac}{t^-} + \yfr{\svar{Sd}{t^-}}{t}\svar{Ipnr}{t^-}\svar{Nt}{t^-} \\
			\svar{Fac}{t^+} &= \begin{cases} \svar{Fac}{t^-} + \yfr{\svar{Sd}{t^-}}{t}\svar{Nt}{t^-}\attr{FER} & \text{if} \quad \attr{FEB}=\text{'N'} \\
					\frac{\yfr{t^{FP-}}{t}}{\yfr{t^{FP-}}{t^{FP+}}}\sgn\attr{FER} & \text{else} \end{cases} \\
			\svar{Nt}{t^+} &= \svar{Nt}{t^-} - \obs{rf}{\attr{OPMO}}{t} \\
			\svar{Sd}{t^+} &= t \end{aligned}$} \par
		with\par
		{$\begin{aligned}
			t^{FP-} &= \sup t \in \vec{t}^{FP}\mid t<t_0 \\
			t^{FP+} &= \inf t \in \vec{t}^{FP}\mid t>t_0 \end{aligned}$} \\
	\hline
	PY & {$\begin{aligned}
			&\obs{rf}{\attr{CURS}}{t}\sgn\attr{PYRT} &\text{if} \quad \attr{PYTP}=\text{'A'}\\
			&c\attr{PYRT} &\text{if} \quad \attr{PYTP}=\text{'N'}\\
			&c\max(0,\svar{Ipnr}{t^-}-\obs{rf}{\attr{RRMO}}{t}) &\text{if} \quad \attr{PYTP}=\text{'I'} \end{aligned}$}\par
		with\par
		$c=\obs{rf}{\attr{CURS}}{t}\sgn\yfr{\svar{Sd}{t^-}}{t}\svar{Nt}{t^-}$
		& {$\begin{aligned}
			\svar{Ipac}{t^+} &= \svar{Ipac}{t^-} + \yfr{\svar{Sd}{t^-}}{t}\svar{Ipnr}{t^-}\svar{Nt}{t^-} \\
			\svar{Fac}{t^+} &= \begin{cases} \svar{Fac}{t^-} + \yfr{\svar{Sd}{t^-}}{t}\svar{Nt}{t^-}\attr{FER} & \text{if} \quad \attr{FEB}=\text{'N'} \\
					\frac{\yfr{t^{FP-}}{t}}{\yfr{t^{FP-}}{t^{FP+}}}\sgn\attr{FER} & \text{else} \end{cases} \\
			\svar{Sd}{t^+} &= t \end{aligned}$} \par
		with\par
		{$\begin{aligned}
			t^{FP-} &= \sup t \in \vec{t}^{FP}\mid t<t_0 \\
			t^{FP+} &= \inf t \in \vec{t}^{FP}\mid t>t_0 \end{aligned}$} \\
	\hline
	FP & {$\begin{aligned}
			&\sgn c &\text{if} \quad \attr{FEB}=\text{'A'}\\
			&c\yfr{\svar{Sd}{t^-}}{t}\svar{Nt}{t^-}+\svar{Fac}{t^-} &\text{if} \quad \attr{FEB}=\text{'N'} \end{aligned}$}\par
		with\par
		$c=\obs{rf}{\attr{CURS}}{t}\attr{FER}$
		& {$\begin{aligned}
			\svar{Ipac}{t^+} &= \svar{Ipac}{t^-} + \yfr{\svar{Sd}{t^-}}{t}\svar{Ipnr}{t^-}\svar{Nt}{t^-} \\
			\svar{Fac}{t^+} &= 0.0 \\
			\svar{Sd}{t^+} &= t \end{aligned}$} \\
	\hline
  	PRD & $\obs{rf}{\attr{CURS}}{t}\sgn (-1)(\attr{PPRD} + \svar{Ipac}{t^-} +$ \par
		$\qquad\qquad \yfr{\svar{Sd}{t^-}}{t}\svar{Ipnr}{t^-}\svar{Nt}{t^-})$
		& {$\begin{aligned}
			\svar{Ipac}{t^+} &= \svar{Ipac}{t^-} + \yfr{\svar{Sd}{t^-}}{t}\svar{Ipnr}{t^-}\svar{Nt}{t^-} \\
			\svar{Fac}{t^+} &= \begin{cases} \svar{Fac}{t^-} + \yfr{\svar{Sd}{t^-}}{t}\svar{Nt}{t^-}\attr{FER} & \text{if} \quad \attr{FEB}=\text{'N'} \\
					\frac{\yfr{t^{FP-}}{t}}{\yfr{t^{FP-}}{t^{FP+}}}\sgn\attr{FER} & \text{else} \end{cases} \\
			\svar{Sd}{t^+} &= t \end{aligned}$} \par
		with\par
		{$\begin{aligned}
			t^{FP-} &= \sup t \in \vec{t}^{FP}\mid t<t_0 \\
			t^{FP+} &= \inf t \in \vec{t}^{FP}\mid t>t_0 \end{aligned}$} \\
	\hline
  	TD & $\obs{rf}{\attr{CURS}}{t}\sgn (\attr{PTD} + \svar{Ipac}{t^-} +$ \par
		$\qquad\qquad \yfr{\svar{Sd}{t^-}}{t}\svar{Ipnr}{t^-}\svar{Nt}{t^-})$
		& {$\begin{aligned}
			\svar{Nt}{t^+} &= 0.0 \\
			\svar{Ipac}{t^+} &= 0.0 \\
			\svar{Fac}{t^+} &= 0.0 \\
			\svar{Ipnr}{t^+} &= 0.0 \\
			\svar{Sd}{t^+} &= t \end{aligned}$} \\
	\hline
	IP & $\obs{rf}{\attr{CURS}}{t}\svar{Isc}{t^-}(\svar{Ipac}{t^-} +$\par
		$\qquad\qquad \yfr{\svar{Sd}{t^-}}{t}\svar{Ipnr}{t^-}\svar{Nt}{t^-})$
		& {$\begin{aligned}
			\svar{Ipac}{t^+} &= 0.0 \\
			\svar{Fac}{t^+} &= \begin{cases} \svar{Fac}{t^-} + \yfr{\svar{Sd}{t^-}}{t}\svar{Nt}{t^-}\attr{FER} & \text{if} \quad \attr{FEB}=\text{'N'} \\
					\frac{\yfr{t^{FP-}}{t}}{\yfr{t^{FP-}}{t^{FP+}}}\sgn\attr{FER} & \text{else} \end{cases} \\
			\svar{Sd}{t^+} &= t \end{aligned}$} \par
		with\par
		{$\begin{aligned}
			t^{FP-} &= \sup t \in \vec{t}^{FP}\mid t<t_0 \\
			t^{FP+} &= \inf t \in \vec{t}^{FP}\mid t>t_0 \end{aligned}$} \\
	\hline
	IPCI & 0.0
		& {$\begin{aligned}
			\svar{Nt}{t^+} &= \svar{Nt}{t^-} + \svar{Ipac}{t^-} + \yfr{\svar{Sd}{t^-}}{t}\svar{Nt}{t^-}\svar{Ipnr}{t^-}\\
			\svar{Ipac}{t^+} &= 0.0 \\
			\svar{Fac}{t^+} &= \begin{cases} \svar{Fac}{t^-} + \yfr{\svar{Sd}{t^-}}{t}\svar{Nt}{t^-}\attr{FER} & \text{if} \quad \attr{FEB}=\text{'N'} \\
					\frac{\yfr{t^{FP-}}{t}}{\yfr{t^{FP-}}{t^{FP+}}}\sgn\attr{FER} & \text{else} \end{cases} \\
			\svar{Sd}{t^+} &= t \end{aligned}$} \par
		with\par
		{$\begin{aligned}
			t^{FP-} &= \sup t \in \vec{t}^{FP}\mid t<t_0 \\
			t^{FP+} &= \inf t \in \vec{t}^{FP}\mid t>t_0 \end{aligned}$} \\
	\hline
	RR & 0.0
		& {$\begin{aligned}
			\svar{Ipac}{t^+} &= \svar{Ipac}{t^-} + \yfr{\svar{Sd}{t^-}}{t}\svar{Ipnr}{t^-}\svar{Nt}{t^-} \\
			\svar{Fac}{t^+} &= \begin{cases} \svar{Fac}{t^-} + \yfr{\svar{Sd}{t^-}}{t}\svar{Nt}{t^-}\attr{FER} & \text{if} \quad \attr{FEB}=\text{'N'} \\
					\frac{\yfr{t^{FP-}}{t}}{\yfr{t^{FP-}}{t^{FP+}}}\sgn\attr{FER} & \text{else} \end{cases} \\
			\svar{Ipnr}{t^+} &= \min(\max(\svar{Ipnr}{t^-}+\Delta r,\attr{RRLF}),\attr{RRLC}) \\
			\svar{Sd}{t^+} &= t \end{aligned}$} \par
		with\par
		{$\begin{aligned}
			\Delta r &= \min(\max(\obs{rf}{\attr{RRMO}}{t}\attr{RRMT}+\attr{RRSP} - \svar{Ipnr}{t^-},\attr{RRPF}),\attr{RRPC}) \\
			t^{FP-} &= \sup t \in \vec{t}^{FP}\mid t<t_0 \\
			t^{FP+} &= \inf t \in \vec{t}^{FP}\mid t>t_0 \end{aligned}$} \\
	\hline
	RRF & 0.0
		& {$\begin{aligned}
			\svar{Ipac}{t^+} &= \svar{Ipac}{t^-} + \yfr{\svar{Sd}{t^-}}{t}\svar{Ipnr}{t^-}\svar{Nt}{t^-} \\
			\svar{Fac}{t^+} &= \begin{cases} \svar{Fac}{t^-} + \yfr{\svar{Sd}{t^-}}{t}\svar{Nt}{t^-}\attr{FER} & \text{if} \quad \attr{FEB}=\text{'N'} \\
					\frac{\yfr{t^{FP-}}{t}}{\yfr{t^{FP-}}{t^{FP+}}}\sgn\attr{FER} & \text{else} \end{cases} \\
			\svar{Ipnr}{t^+} &= \attr{RRNXT} \\
			\svar{Sd}{t^+} &= t \end{aligned}$} \par
		with\par
		{$\begin{aligned}
			t^{FP-} &= \sup t \in \vec{t}^{FP}\mid t<t_0 \\
			t^{FP+} &= \inf t \in \vec{t}^{FP}\mid t>t_0 \end{aligned}$} \\
	\hline
	SC & 0.0
		& {$\begin{aligned}
			\svar{Ipac}{t^+} &= \svar{Ipac}{t^-} + \yfr{\svar{Sd}{t^-}}{t}\svar{Ipnr}{t^-}\svar{Nt}{t^-} \\
			\svar{Fac}{t^+} &= \begin{cases} \svar{Fac}{t^-} + \yfr{\svar{Sd}{t^-}}{t}\svar{Nt}{t^-}\attr{FER} & \text{if} \quad \attr{FEB}=\text{'N'} \\
					\frac{\yfr{t^{FP-}}{t}}{\yfr{t^{FP-}}{t^{FP+}}}\sgn\attr{FER} & \text{else} \end{cases} \\
			\svar{Nsc}{t^+} &= \begin{cases} \svar{Nsc}{t^-} & \text{if} \quad \attr{SCEF} = [x]0[x] \\
							\frac{\obs{rf}{\attr{SCMO}}{t} - \attr{SCIED}}{\attr{SCIED}} & \text{else} \end{cases} \\
			\svar{Isc}{t^+} &= \begin{cases} \svar{Isc}{t^-} & \text{if} \quad \attr{SCEF} = 0[x][x] \\
							\frac{\obs{rf}{\attr{SCMO}}{t} - \attr{SCIED}}{\attr{SCIED}} & \text{else} \end{cases} \\
			\svar{Sd}{t^+} &= t \end{aligned}$} \par
		with\par
		{$\begin{aligned}
			t^{FP-} &= \sup t \in \vec{t}^{FP}\mid t<t_0 \\
			t^{FP+} &= \inf t \in \vec{t}^{FP}\mid t>t_0 \end{aligned}$} \\
	\hline
	CE & 0.0 & \stf{AD}{PAM} \\
\end{functions}



%%%%%%%%%%%%%%%%%%%% subsection: lam %%%%%%%%%%%%%%%%%%%%

\subsection{LAM: Linear Amortizer}\label{sec:lam}


% ---------------------- table: lam schedule ----------------------
\begin{schedule}{LAM}
	AD & & Same as PAM \\
	\hline
	IED & & Same as PAM \\
	\hline
	PR & $t^{PR} = \sdl{s}{\attr{PRCL}}{T^{MD}}$
		& with\par $s = \begin{cases} \undef & \text{if} \quad \attr{PRANX}=\undef \land \attr{PRCL}=\undef\\
					   \attr{IED}+\attr{PRCL} & \text{else if} \quad \attr{PRANX} = \undef \\
					   \attr{PRANX} & \text{else} \end{cases}$ \\
	\hline
	PP & & Same as PAM \\
	\hline
	PY & & Same as PAM \\
	\hline
	FP & & Same as PAM \\
	\hline
	PRD & & Same as PAM \\
	\hline
	TD & & Same as PAM \\
	\hline
	IP & & Same as PAM \\
	\hline
	IPCI & & Same as PAM \\
  	\hline
	IPCB & $\vec{t}^{IPCB} = \begin{cases} \undef & \text{if} \quad \attr{IPCB}\neq\text{'NTL'} \\
					\sdl{s}{\attr{IPCBCL}}{T^{MD}} & \text{else} \end{cases}$
		& with\par $s = \begin{cases} \undef & \text{if} \quad \attr{IPCBANX}=\undef \land \attr{IPCBCL}=\undef\\
					   \attr{IED}+\attr{IPCBCL} & \text{else if} \quad \attr{IPCBANX} = \undef \\
					   \attr{IPCBANX} & \text{else} \end{cases}$ \\
	\hline
	RR & & Same as PAM \\
	\hline
	RRF & & Same as PAM \\
  	\hline
	SC & & Same as PAM \\
	\hline
	CE & & Same as PAM \\
\end{schedule}


% ---------------------- table: lam states ----------------------
\begin{states}{LAM}
	$\svar{Tmd}{}$ & $\svar{Tmd}{t_0}=\begin{cases}
						\attr{MD} & \text{if} \attr{MD}\neq\undef\\
						t^-+ceil(\frac{\attr{NT}}{\attr{PRNXT}})\attr{PRCL}
							\end{cases}$ & where\par
						$t^- = \begin{cases} \attr{PRANX} & \text{if}\quad \attr{PRANX}\neq\undef \land \attr{PRANX}\geq t_0 \\
								\attr{IED}+\attr{PRCL} & \text{else if}\quad \attr{IED}+\attr{PRCL}\geq t_0 \\
								\sup t \in \vec{t}^{PR}\mid t<t_0 & \text{else} \end{cases}$ \\
	\hline
  	$\svar{Nt}{}$ & & Same as PAM \\
	\hline
	$\svar{Ipnr}{}$ & & Same as PAM \\
  	\hline
  	$\svar{Ipac}{}$ & & Same as PAM \\
	\hline
  	$\svar{Fac}{}$ & & Same as PAM \\
  	\hline
  	$\svar{Nsc}{}$ & & Same as PAM \\
  	\hline
  	$\svar{Isc}{}$ & & Same as PAM \\
	\hline
  	$\svar{Prf}{}$ & & Same as PAM \\
	\hline
	$\svar{Sd}{}$ & & Same as PAM \\
	\hline
	$\svar{Prnxt}{}$ & $\svar{Prnxt}{t_0} = \begin{cases} \attr{PRNXT} & \text{if}\quad \attr{PRNXT}\neq\undef \\
					\attr{NT}(ceil(\frac{\yfr{s}{T^{MD}}}{\yfr{s}{s+\attr{PRCL}}}))^{-1} & \text{else} \end{cases}$
		& with\par $s = \begin{cases}
					\attr{PRANX} & \text{if}\quad \attr{PRANX}\neq\undef \land \attr{PRANX}>t_0\\
					\attr{IED}+\attr{PRCL} & \text{else if}\quad \attr{PRANX} = \undef \land \attr{IED}+\attr{PRCL}>t_0 \\
					t^- & \text{else} \end{cases}$\par
		and where $t^- = \sup t \in \vec{t}^{PR}\mid t<t_0$ \\
	\hline
	$\svar{Ipcb}{}$ & $\svar{Ipcb}{t_0}= \begin{cases} 0.0 & \text{if}\quad t_0<\attr{IED} \\
					\sgn\attr{NT} & \text{else if}\quad \attr{IPCB}=\text{'NT'} \\
					\sgn\attr{IPCBA} & \text{else} \end{cases}$ & \\
\end{states}


% ---------------------- table: lam functions ----------------------
\begin{functions}{LAM}
	AD & \pof{AD}{PAM} & \stf{AD}{PAM} \\
	\hline
	IED & \pof{IED}{PAM}
		& {$\begin{aligned}
			\svar{Nt}{t^+} &=\sgn\attr{NT} \\
			\svar{Ipnr}{t^+} &= \attr{IPNR} \\
			\svar{Ipac}{t^+} &= \begin{cases} \attr{IPAC} & \text{if} \quad \attr{IPAC} \neq \undef \\
							y\svar{Nt}{t^+}\svar{Ipnr}{t^+} & \text{if} \quad \attr{IPANX} \neq \undef \land \attr{IPANX}<t \\
							0.0 & \text{else} \end{cases} \\
			\svar{Sd}{t^+} &= t \\
			\svar{Ipcb}{t^+} &= \begin{cases} \sgn\attr{NT} & \text{if}\quad \attr{IPCB}=\text{'NT'} \\
							\sgn\attr{IPCBA} & \text{else} \end{cases} \end{aligned}$} \\
	\hline
	PR & $\obs{rf}{\attr{CURS}}{t}\sgn\svar{Nsc}{t^-}\svar{Prnxt}{t^-}$
		& {$\begin{aligned}
			\svar{Nt}{t^+} &= \svar{Nt}{t^-}-\sgn\svar{Prnxt}{t^-} \\
			\svar{Fac}{t^+} &= \begin{cases} \svar{Fac}{t^-} + \yfr{\svar{Sd}{t^-}}{t}\svar{Nt}{t^-}\attr{FER} & \text{if} \quad \attr{FEB}=\text{'N'} \\
					\frac{\yfr{t^{FP-}}{t}}{\yfr{t^{FP-}}{t^{FP+}}}\sgn\attr{FER} & \text{else} \end{cases} \\
			\svar{Ipcb}{t^+} &= \begin{cases} \svar{Ipcb}{t^-} & \text{if}\quad \attr{IPCB}\neq\text{'NT'} \\
							\svar{Nt}{t^+} & \text{else} \end{cases}\\
			\svar{Sd}{t^+} &= t \end{aligned}$} \par
		with\par
		{$\begin{aligned}
			t^{FP-} &= \sup t \in \vec{t}^{FP}\mid t<t_0 \\
			t^{FP+} &= \inf t \in \vec{t}^{FP}\mid t>t_0 \end{aligned}$} \\
	\hline
	PP & \pof{PP}{PAM}
		& {$\begin{aligned}
			\svar{Ipac}{t^+} &= \svar{Ipac}{t^-} + \yfr{\svar{Sd}{t^-}}{t}\svar{Ipnr}{t^-}\svar{Ipcb}{t^-} \\
			\svar{Fac}{t^+} &= \begin{cases} \svar{Fac}{t^-} + \yfr{\svar{Sd}{t^-}}{t}\svar{Nt}{t^-}\attr{FER} & \text{if} \quad \attr{FEB}=\text{'N'} \\
					\frac{\yfr{t^{FP-}}{t}}{\yfr{t^{FP-}}{t^{FP+}}}\sgn\attr{FER} & \text{else} \end{cases} \\
			\svar{Nt}{t^+} &= \svar{Nt}{t^-} - \obs{rf}{\attr{OPMO}}{t} \\
			\svar{Ipcb}{t^+} &= \begin{cases} \svar{Ipcb}{t^-} & \text{if}\quad \attr{IPCB}\neq\text{'NT'} \\
							\svar{Nt}{t^+} & \text{else} \end{cases}\\
			\svar{Sd}{t^+} &= t \end{aligned}$} \par
		with\par
		{$\begin{aligned}
			t^{FP-} &= \sup t \in \vec{t}^{FP}\mid t<t_0 \\
			t^{FP+} &= \inf t \in \vec{t}^{FP}\mid t>t_0 \end{aligned}$} \\
	\hline
	PY & \pof{PY}{PAM}
		& {$\begin{aligned}
			\svar{Ipac}{t^+} &= \svar{Ipac}{t^-} + \yfr{\svar{Sd}{t^-}}{t}\svar{Ipnr}{t^-}\svar{Ipcb}{t^-} \\
			\svar{Fac}{t^+} &= \begin{cases} \svar{Fac}{t^-} + \yfr{\svar{Sd}{t^-}}{t}\svar{Nt}{t^-}\attr{FER} & \text{if} \quad \attr{FEB}=\text{'N'} \\
					\frac{\yfr{t^{FP-}}{t}}{\yfr{t^{FP-}}{t^{FP+}}}\sgn\attr{FER} & \text{else} \end{cases} \\
			\svar{Sd}{t^+} &= t \end{aligned}$} \par
		with\par
		{$\begin{aligned}
			t^{FP-} &= \sup t \in \vec{t}^{FP}\mid t<t_0 \\
			t^{FP+} &= \inf t \in \vec{t}^{FP}\mid t>t_0 \end{aligned}$} \\
	\hline
	FP & \pof{FP}{PAM}
		& {$\begin{aligned}
			\svar{Ipac}{t^+} &= \svar{Ipac}{t^-} + \yfr{\svar{Sd}{t^-}}{t}\svar{Ipnr}{t^-}\svar{Ipcb}{t^-} \\
			\svar{Fac}{t^+} &= 0.0 \\
			\svar{Sd}{t^+} &= t \end{aligned}$} \\
	\hline
  	PRD & $\obs{rf}{\attr{CURS}}{t}\sgn (-1)(\attr{PPRD} + \svar{Ipac}{t^-} +$ \par
		$\qquad\qquad \yfr{\svar{Sd}{t^-}}{t}\svar{Ipnr}{t^-}\svar{Ipcb}{t^-})$
		& {$\begin{aligned}
			\svar{Ipac}{t^+} &= \svar{Ipac}{t^-} + \yfr{\svar{Sd}{t^-}}{t}\svar{Ipnr}{t^-}\svar{Ipcb}{t^-} \\
			\svar{Fac}{t^+} &= \begin{cases} \svar{Fac}{t^-} + \yfr{\svar{Sd}{t^-}}{t}\svar{Nt}{t^-}\attr{FER} & \text{if} \quad \attr{FEB}=\text{'N'} \\
					\frac{\yfr{t^{FP-}}{t}}{\yfr{t^{FP-}}{t^{FP+}}}\sgn\attr{FER} & \text{else} \end{cases} \\
			\svar{Sd}{t^+} &= t \end{aligned}$} \par
		with\par
		{$\begin{aligned}
			t^{FP-} &= \sup t \in \vec{t}^{FP}\mid t<t_0 \\
			t^{FP+} &= \inf t \in \vec{t}^{FP}\mid t>t_0 \end{aligned}$} \\
	\hline
  	TD & $\obs{rf}{\attr{CURS}}{t}\sgn (\attr{PTD} + \svar{Ipac}{t^-} +$ \par
		$\qquad\qquad \yfr{\svar{Sd}{t^-}}{t}\svar{Ipnr}{t^-}\svar{Ipcb}{t^-})$
		& \stf{TD}{PAM} \\
	\hline
	IP & $\obs{rf}{\attr{CURS}}{t}\svar{Isc}{t^-}(\svar{Ipac}{t^-} +$ \par
		$\qquad\qquad \yfr{\svar{Sd}{t^-}}{t}\svar{Ipnr}{t^-}\svar{Ipcb}{t^-})$
		& \stf{IP}{PAM} \\
	\hline
	IPCI & \pof{IPCI}{PAM}
		& {$\begin{aligned}
			\svar{Nt}{t^+} &= \svar{Nt}{t^-} + \svar{Ipac}{t^-} + \yfr{\svar{Sd}{t^-}}{t}\svar{Ipnr}{t^-}\svar{Ipcb}{t^-}\\
			\svar{Ipac}{t^+} &= 0.0 \\
			\svar{Fac}{t^+} &= \begin{cases} \svar{Fac}{t^-} + \yfr{\svar{Sd}{t^-}}{t}\svar{Nt}{t^-}\attr{FER} & \text{if} \quad \attr{FEB}=\text{'N'} \\
					\frac{\yfr{t^{FP-}}{t}}{\yfr{t^{FP-}}{t^{FP+}}}\sgn\attr{FER} & \text{else} \end{cases} \\
			\svar{Ipcb}{t^+} &= \begin{cases} \svar{Ipcb}{t^-} & \text{if}\quad \attr{IPCB}\neq\text{'NT'} \\
							\svar{Nt}{t^+} & \text{else} \end{cases}\\
			\svar{Sd}{t^+} &= t \end{aligned}$} \par
		with\par
		{$\begin{aligned}
			t^{FP-} &= \sup t \in \vec{t}^{FP}\mid t<t_0 \\
			t^{FP+} &= \inf t \in \vec{t}^{FP}\mid t>t_0 \end{aligned}$} \\
	\hline
	IPCB & 0.0
		& {$\begin{aligned}
			\svar{Ipcb}{t^+} &= \svar{Nt}{t^-} \\
			\svar{Ipac}{t^+} &= \svar{Ipac}{t^-} + \yfr{\svar{Sd}{t^-}}{t}\svar{Ipnr}{t^-}\svar{Ipcb}{t^-} \\
			\svar{Fac}{t^+} &= \begin{cases} \svar{Fac}{t^-} + \yfr{\svar{Sd}{t^-}}{t}\svar{Nt}{t^-}\attr{FER} & \text{if} \quad \attr{FEB}=\text{'N'} \\
					\frac{\yfr{t^{FP-}}{t}}{\yfr{t^{FP-}}{t^{FP+}}}\sgn\attr{FER} & \text{else} \end{cases} \\
			\svar{Sd}{t^+} &= t \end{aligned}$} \par
		with\par
		{$\begin{aligned}
			t^{FP-} &= \sup t \in \vec{t}^{FP}\mid t<t_0 \\
			t^{FP+} &= \inf t \in \vec{t}^{FP}\mid t>t_0 \end{aligned}$} \\
	\hline
	RR & \pof{RR}{PAM}
		& {$\begin{aligned}
			\svar{Ipac}{t^+} &= \svar{Ipac}{t^-} + \yfr{\svar{Sd}{t^-}}{t}\svar{Ipnr}{t^-}\svar{Ipcb}{t^-} \\
			\svar{Fac}{t^+} &= \begin{cases} \svar{Fac}{t^-} + \yfr{\svar{Sd}{t^-}}{t}\svar{Nt}{t^-}\attr{FER} & \text{if} \quad \attr{FEB}=\text{'N'} \\
					\frac{\yfr{t^{FP-}}{t}}{\yfr{t^{FP-}}{t^{FP+}}}\sgn\attr{FER} & \text{else} \end{cases} \\
			\svar{Ipnr}{t^+} &= \min(\max(\svar{Ipnr}{t^-}+\Delta r,\attr{RRLF}),\attr{RRLC}) \\
			\svar{Sd}{t^+} &= t \end{aligned}$}\par
		with\par
		{$\begin{aligned}
			\Delta r &= \min(\max(\obs{rf}{\attr{RRMO}}{t}\attr{RRMT}+\attr{RRSP} - \svar{Ipnr}{t^-},\attr{RRPF}),\attr{RRPC}) \\
			t^{FP-} &= \sup t \in \vec{t}^{FP}\mid t<t_0 \\
			t^{FP+} &= \inf t \in \vec{t}^{FP}\mid t>t_0 \end{aligned}$} \\
	\hline
	RRF & \pof{RRF}{PAM}
		& {$\begin{aligned}
			\svar{Ipac}{t^+} &= \svar{Ipac}{t^-} + \yfr{\svar{Sd}{t^-}}{t}\svar{Ipnr}{t^-}\svar{Ipcb}{t^-} \\
			\svar{Fac}{t^+} &= \begin{cases} \svar{Fac}{t^-} + \yfr{\svar{Sd}{t^-}}{t}\svar{Nt}{t^-}\attr{FER} & \text{if} \quad \attr{FEB}=\text{'N'} \\
					\frac{\yfr{t^{FP-}}{t}}{\yfr{t^{FP-}}{t^{FP+}}}\sgn\attr{FER} & \text{else} \end{cases} \\
			\svar{Ipnr}{t^+} &= \attr{RRNXT} \\
			\svar{Sd}{t^+} &= t \end{aligned}$} \par
		with\par
		{$\begin{aligned}
			t^{FP-} &= \sup t \in \vec{t}^{FP}\mid t<t_0 \\
			t^{FP+} &= \inf t \in \vec{t}^{FP}\mid t>t_0 \end{aligned}$} \\
	\hline
	SC & \pof{SC}{PAM}
		& {$\begin{aligned}
			\svar{Ipac}{t^+} &= \svar{Ipac}{t^-} + \yfr{\svar{Sd}{t^-}}{t}\svar{Ipnr}{t^-}\svar{Ipcb}{t^-} \\
			\svar{Fac}{t^+} &= \begin{cases} \svar{Fac}{t^-} + \yfr{\svar{Sd}{t^-}}{t}\svar{Nt}{t^-}\attr{FER} & \text{if} \quad \attr{FEB}=\text{'N'} \\
					\frac{\yfr{t^{FP-}}{t}}{\yfr{t^{FP-}}{t^{FP+}}}\sgn\attr{FER} & \text{else} \end{cases} \\
			\svar{Nsc}{t^+} &= \begin{cases} \svar{Nsc}{t^-} & \text{if} \quad \attr{SCEF} = [x]0[x] \\
							\frac{\obs{rf}{\attr{SCMO}}{t} - \attr{SCIED}}{\attr{SCIED}} & \text{else} \end{cases} \\
			\svar{Isc}{t^+} &= \begin{cases} \svar{Isc}{t^-} & \text{if} \quad \attr{SCEF} = 0[x][x] \\
							\frac{\obs{rf}{\attr{SCMO}}{t} - \attr{SCIED}}{\attr{SCIED}} & \text{else} \end{cases} \\
			\svar{Sd}{t^+} &= t \end{aligned}$} \par
		with\par
		{$\begin{aligned}
			t^{FP-} &= \sup t \in \vec{t}^{FP}\mid t<t_0 \\
			t^{FP+} &= \inf t \in \vec{t}^{FP}\mid t>t_0 \end{aligned}$} \\
	\hline
	CE & \pof{CE}{PAM} & \stf{AD}{PAM} \\
\end{functions}



%%%%%%%%%%%%%%%%%%%% subsection: lax %%%%%%%%%%%%%%%%%%%%

\subsection{LAX: Exotic Linear Amortizer}\label{sec:lax}


% ---------------------- table: lax schedule ----------------------
\begin{schedule}{LAX}
	AD & & Same as PAM \\
	\hline
	IED & & Same as PAM \\
	\hline
	PR & $\vec{t}^{PR} = \begin{cases} \{ t_1, t_2, ..., t_i, ... \} & \text{if}\quad \attr{ARPRCL}=\undef \\
					s_1 \cup s_2 \cup ... \cup s_i \cup ... & \text{else} \end{cases}$ \par
		with\par
		$s_i=\sdl{\attr{ARPRANX}_i}{\vec{C}_i}{\attr{ARPRANX}_{i+1}}$, $i\in\{1,2,...,\mid\attr{ARINCDEC}\mid\} \mid \attr{ARINCDEC}_i = \text{'DEC'}$
		& with\par $\vec{C} = \begin{cases} \attr{ARPRCL} & \text{if} \quad \mid\attr{ARPRCL}\mid = \mid \attr{ARPRANX}\mid \\
				   \{ c_1, c_2, ..., c_n \}  & \text{else} \end{cases}$ \par
			where\par
			$n=\mid\attr{ARPRANX}\mid, c_k=\attr{ARPRCL}_1\forall k$ \\
	\hline
	PI & $\vec{t}^{PI} = \begin{cases} \{ t_1, t_2, ..., t_i, ... \} & \text{if}\quad \attr{ARPRCL}=\undef \\
					s_1 \cup s_2 \cup ... \cup s_i \cup ... & \text{else} \end{cases}$ \par
		with\par
		$s_i=\sdl{\attr{ARPRANX}_i}{\vec{C}_i}{\attr{ARPRANX}_{i+1}}$, $i\in\{1,2,...,\mid\attr{ARINCDEC}\mid\} \mid \attr{ARINCDEC}_i = \text{'INC'}$
		& with\par $\vec{C} = \begin{cases} \attr{ARPRCL} & \text{if} \quad \mid\attr{ARPRCL}\mid = \mid \attr{ARPRANX}\mid \\
				   \{ c_1, c_2, ..., c_n \}  & \text{else} \end{cases}$ \par
			where\par
			$n=\mid\attr{ARPRANX}\mid, c_k=\attr{ARPRCL}_1\forall k$ \\
	\hline
	PRF & $\vec{t}^{PRF} = \attr{ARPRANX}$ & \\
	\hline
	PP & & Same as PAM \\
	\hline
	PY & & Same as PAM \\
	\hline
	FP & & Same as PAM \\
	\hline
	PRD & & Same as PAM \\
	\hline
	TD & & Same as PAM \\
	\hline
	IP & $\vec{t}^{IP} = \vsdl{\attr{ARIPANX}}{\attr{ARIPCL}}{\svar{Tmd}{t_0}}$ & \\
	\hline
	IPCI & & Same as PAM \\
  	\hline
	IPCB & & Same as LAM \\
	\hline
	RR & $\vec{t}^{RR} = \begin{cases} \{ t_1, t_2, ..., t_i, ... \} & \text{if}\quad \attr{ARRRCL}=\undef \\
					s_1 \cup s_2 \cup ... \cup s_i \cup ... & \text{else} \end{cases}$ \par
		with\par
		$s_i=\sdl{\attr{ARRRANX}_i}{\vec{C}_i}{\attr{ARRRANX}_{i+1}}$, $i\in\{1,2,...,\mid\attr{ARFIXVAR}\mid\} \mid \attr{ARFIXVAR}_i = \text{'V'}$
		& with\par $\vec{C} = \begin{cases} \attr{ARRRCL} & \text{if} \quad \mid\attr{ARRRCL}\mid = \mid \attr{ARRRANX}\mid \\
				   \{ c_1, c_2, ..., c_n \}  & \text{else} \end{cases}$ \par
			where\par
			$n=\mid\attr{ARRRANX}\mid, c_k=\attr{ARRRCL}_1\forall k$ \\
	\hline
	RRF & $\vec{t}^{RRF} = \begin{cases} \{ t_1, t_2, ..., t_i, ... \} & \text{if}\quad \attr{ARRRCL}=\undef \\
					s_1 \cup s_2 \cup ... \cup s_i \cup ... & \text{else} \end{cases}$ \par
		with\par
		$s_i=\sdl{\attr{ARRRANX}_i}{\vec{C}_i}{\attr{ARRRANX}_{i+1}}$, $i\in\{1,2,...,\mid\attr{ARFIXVAR}\mid\} \mid \attr{ARFIXVAR}_i = \text{'F'}$
		& with\par $\vec{C} = \begin{cases} \attr{ARRRCL} & \text{if} \quad \mid\attr{ARRRCL}\mid = \mid \attr{ARRRANX}\mid \\
				   \{ c_1, c_2, ..., c_n \}  & \text{else} \end{cases}$ \par
			where\par
			$n=\mid\attr{ARRRANX}\mid, c_k=\attr{ARRRCL}_1\forall k$ \\
	\hline
	SC & & Same as PAM \\
	\hline
	CE & & Same as PAM \\
\end{schedule}


% ---------------------- table: lax states ----------------------
\begin{states}{LAX}
	$\svar{Tmd}{}$ & $\svar{Tmd}{t_0}=\begin{cases}
						\attr{MD} & \text{if}\quad \attr{MD}\neq\undef\\
						\inf t>t_0 \mid N(t)=0 & \text{else} \end{cases}$ \par
	with\par
	$N(t) = \attr{NT} + \sum_{i=1}^{n(t)} (-1)^k \attr{ARPRNXT}_i \mid s_i \mid$ &  where\par
	{$\begin{aligned}
		n(t) &= \begin{cases} \sup k\in\Nat \mid \attr{ARPRANX}_k<t & \text{if}\quad t<max(\attr{ARPRANX}) \\
					\mid\attr{ARPRANX}\mid & \text{else} \end{cases} \\
		k &= \begin{cases} 0 & \text{if}\quad \attr{ARINCDEC}_i=\text{'INC'}\\ 1 & \text{else} \end{cases} \\
		s_i &= \begin{cases} \{ \attr{ARPRANX}_i \} & \text{if}\quad \attr{ARPRCL} = \undef \\
					\sdl{\attr{ARPRANX}_i}{\vec{C}_i}{T_i} & \text{else} \end{cases} \\
		T_i &= \begin{cases} \attr{ARPRANX}_{i+1} & \text{if}\quad i<\mid\attr{ARPRANX}\mid \\
					t & \text{else} \end{cases} \\
		\vec{C} &= \begin{cases} \attr{ARRRCL} & \text{if} \quad \mid\attr{ARRRCL}\mid = \mid \attr{ARRRANX}\mid \\
				   \{ c_1, c_2, ..., c_n \}  & \text{else} \end{cases}
	\end{aligned}$} \\
	\hline
  	$\svar{Nt}{}$ & & Same as PAM \\
	\hline
	$\svar{Ipnr}{}$ & & Same as PAM \\
  	\hline
  	$\svar{Ipac}{}$ & & Same as PAM \\
	\hline
  	$\svar{Fac}{}$ & & Same as PAM \\
  	\hline
  	$\svar{Nsc}{}$ & & Same as PAM \\
  	\hline
  	$\svar{Isc}{}$ & & Same as PAM \\
	\hline
  	$\svar{Prf}{}$ & & Same as PAM \\
	\hline
	$\svar{Sd}{}$ & & Same as PAM \\
	\hline
	$\svar{Prnxt}{}$ & $\svar{Prnxt}{t_0} = \begin{cases} 0.0 & \text{if}\quad t_0\geq\attr{ARPRANX}_1 \\
							\attr{ARPRNXT}_i & \text{else} \end{cases}$ & where\par
						$i = \sup k\in\Nat \mid \attr{ARPRANX}_k<t_0$  \\
	\hline
	$\svar{Ipcb}{}$ & & Same as LAM \\
\end{states}


% ---------------------- table: lax functions ----------------------
\begin{functions}{LAX}
	AD & \pof{AD}{PAM} & \stf{AD}{PAM} \\
	\hline
	IED & \pof{IED}{PAM} & \stf{IED}{LAM} \\
	\hline
	PR & \pof{PR}{LAM} & \stf{PR}{LAM} \\
	\hline
	PI & $\obs{rf}{\attr{CURS}}{t}\sgn(-1)\svar{Nsc}{t^-}\svar{Prnxt}{t^-}$ & {$\begin{aligned}
			\svar{Nt}{t^+} &= \svar{Nt}{t^-}+\sgn\svar{Prnxt}{t^-} \\
			\svar{Ipac}{t^+} &= \svar{Ipac}{t^-} + \yfr{\svar{Sd}{t^-}}{t}\svar{Ipnr}{t^-}\svar{Ipcb}{t^-} \\
			\svar{Fac}{t^+} &= \begin{cases} \svar{Fac}{t^-} + \yfr{\svar{Sd}{t^-}}{t}\svar{Nt}{t^-}\attr{FER} & \text{if} \quad \attr{FEB}=\text{'N'} \\
					\frac{\yfr{t^{FP-}}{t}}{\yfr{t^{FP-}}{t^{FP+}}}\sgn\attr{FER} & \text{else} \end{cases} \\
			\svar{Ipcb}{t^+} &= \begin{cases} \svar{Ipcb}{t^-} & \text{if}\quad \attr{IPCB}\neq\text{'NT'} \\
							\svar{Nt}{t^+} & \text{else} \end{cases}\\
			\svar{Sd}{t^+} &= t \end{aligned}$} \par
		with\par
		{$\begin{aligned}
			t^{FP-} &= \sup t \in \vec{t}^{FP}\mid t<t_0 \\
			t^{FP+} &= \inf t \in \vec{t}^{FP}\mid t>t_0 \end{aligned}$} \\
	\hline
	PRF & 0.0 & {$\begin{aligned}
			\svar{Prnxt}{t^+} &= \attr{ARPRNXT}_i \\
			\svar{Ipac}{t^+} &= \svar{Ipac}{t^-} + \yfr{\svar{Sd}{t^-}}{t}\svar{Ipnr}{t^-}\svar{Ipcb}{t^-} \\
			\svar{Fac}{t^+} &= \begin{cases} \svar{Fac}{t^-} + \yfr{\svar{Sd}{t^-}}{t}\svar{Nt}{t^-}\attr{FER} & \text{if} \quad \attr{FEB}=\text{'N'} \\
					\frac{\yfr{t^{FP-}}{t}}{\yfr{t^{FP-}}{t^{FP+}}}\sgn\attr{FER} & \text{else} \end{cases} \\
			\svar{Ipcb}{t^+} &= \begin{cases} \svar{Ipcb}{t^-} & \text{if}\quad \attr{IPCB}\neq\text{'NT'} \\
							\svar{Nt}{t^+} & \text{else} \end{cases}\\
			\svar{Sd}{t^+} &= t \end{aligned}$} \par
		with\par
		{$\begin{aligned}
			i &= \sup k\in\Nat \mid \attr{ARPRANX}_k=t \\
			t^{FP-} &= \sup t \in \vec{t}^{FP}\mid t<t_0 \\
			t^{FP+} &= \inf t \in \vec{t}^{FP}\mid t>t_0 \end{aligned}$} \\
	\hline
	PP & \pof{PP}{PAM} & \stf{PP}{LAM} \\
	\hline
	PY & \pof{PY}{PAM} & \stf{PY}{LAM} \\
	\hline
	FP & \pof{FP}{PAM} & \stf{FP}{LAM} \\
	\hline
  	PRD & \pof{PRD}{LAM} & \stf{PRD}{LAM} \\
	\hline
  	TD & \pof{TD}{LAM} & \stf{TD}{PAM} \\
	\hline
	IP & \pof{IP}{LAM} & \stf{IP}{PAM} \\
	\hline
	IPCI & \pof{IPCI}{PAM} & \stf{IPCI}{LAM} \\
	\hline
	IPCB & \pof{IPCB}{LAM} & \stf{IPCB}{LAM} \\
	\hline
	RR & \pof{RR}{PAM}
		& {$\begin{aligned}
			\svar{Ipac}{t^+} &= \svar{Ipac}{t^-} + \yfr{\svar{Sd}{t^-}}{t}\svar{Ipnr}{t^-}\svar{Ipcb}{t^-} \\
			\svar{Fac}{t^+} &= \begin{cases} \svar{Fac}{t^-} + \yfr{\svar{Sd}{t^-}}{t}\svar{Nt}{t^-}\attr{FER} & \text{if} \quad \attr{FEB}=\text{'N'} \\
					\frac{\yfr{t^{FP-}}{t}}{\yfr{t^{FP-}}{t^{FP+}}}\sgn\attr{FER} & \text{else} \end{cases} \\
			\svar{Ipnr}{t^+} &= \min(\max(\svar{Ipnr}{t^-}+\Delta r,\attr{RRLF}),\attr{RRLC}) \\
			\svar{Sd}{t^+} &= t \end{aligned}$}\par
		with\par
		{$\begin{aligned}
			\Delta r &= \min(\max(\obs{rf}{\attr{RRMO}}{t}\attr{RRMT}+\attr{ARRATE}_i - \svar{Ipnr}{t^-},\attr{RRPF}),\attr{RRPC}) \\
			i &= \sup k\in\Nat \mid \attr{ARPRANX}_k=t \\
			t^{FP-} &= \sup t \in \vec{t}^{FP}\mid t<t_0 \\
			t^{FP+} &= \inf t \in \vec{t}^{FP}\mid t>t_0 \end{aligned}$} \\
	\hline
	RRF & \pof{RRF}{PAM}
		& {$\begin{aligned}
			\svar{Ipac}{t^+} &= \svar{Ipac}{t^-} + \yfr{\svar{Sd}{t^-}}{t}\svar{Ipnr}{t^-}\svar{Ipcb}{t^-} \\
			\svar{Fac}{t^+} &= \begin{cases} \svar{Fac}{t^-} + \yfr{\svar{Sd}{t^-}}{t}\svar{Nt}{t^-}\attr{FER} & \text{if} \quad \attr{FEB}=\text{'N'} \\
					\frac{\yfr{t^{FP-}}{t}}{\yfr{t^{FP-}}{t^{FP+}}}\sgn\attr{FER} & \text{else} \end{cases} \\
			\svar{Ipnr}{t^+} &= \attr{ARRATE}_i \\
			\svar{Sd}{t^+} &= t \end{aligned}$} \par
		with\par
		{$\begin{aligned}
			i &= \sup k\in\Nat \mid \attr{ARPRANX}_k=t \\
			t^{FP-} &= \sup t \in \vec{t}^{FP}\mid t<t_0 \\
			t^{FP+} &= \inf t \in \vec{t}^{FP}\mid t>t_0 \end{aligned}$} \\
	\hline
	SC & \pof{SC}{PAM} & \stf{SC}{LAM} \\
	\hline
	CE & \pof{CE}{PAM} & \stf{AD}{PAM} \\
\end{functions}



%%%%%%%%%%%%%%%%%%%% subsection: nam %%%%%%%%%%%%%%%%%%%%

\subsection{NAM: Negative Amortizer}\label{sec:nam}


% ---------------------- table: nam schedule ----------------------
\begin{schedule}{NAM}
	AD & & Same as PAM \\
	\hline
	IED & & Same as PAM \\
	\hline
	PR & & Same as LAM \\
	\hline
	PP & & Same as PAM \\
	\hline
	PY & & Same as PAM \\
	\hline
	FP & & Same as PAM \\
	\hline
	PRD & & Same as PAM \\
	\hline
	TD & & Same as PAM \\
	\hline
	IP & $\vec{t}^{IP} = (\vec{u},\vec{v})$ \par
		where \par
		{$\begin{aligned} \vec{u} &= \begin{cases} \undef & \text{if}\quad \attr{IPANX}=\undef\land\attr{IPCL}=\undef \\
							\undef & \text{if}\quad \attr{IPCED}\neq\undef\land\attr{IPCED}\geq\attr{T}\\
							\sdl{r}{\attr{IPCL}}{T} & \text{else} \end{cases} \\
				\vec{v} &= \sdl{s}{\attr{PRCL}}{T^{MD}} \end{aligned}$}
		 & with\par {$\begin{aligned} r &= \begin{cases} \attr{IPCED} & \text{if}\quad \attr{IPCED}\neq\undef \\
								\attr{IPANX} & \text{else if}\quad \attr{IPANX}\neq\undef \\
								\attr{IED}+\attr{IPCL} & \text{else if}\quad \attr{IPCL}\neq\undef \\
								\undef & \text{else} \end{cases} \\
						T &= s-\attr{PRCL} \\
						s &= \begin{cases} \attr{IED}+\attr{PRCL} & \text{if} \quad \attr{PRANX} = \undef \\
					   \attr{PRANX} & \text{else} \end{cases} \end{aligned}$} \\
	\hline
	IPCI & & Same as PAM \\
  	\hline
	IPCB & & Same as LAM \\
	\hline
	RR & & Same as PAM \\
	\hline
	RRF & & Same as PAM \\
  	\hline
	SC & & Same as PAM \\
	\hline
	CE & & Same as PAM \\
\end{schedule}


% ---------------------- table: nam states ----------------------
\begin{states}{NAM}
	$\svar{Tmd}{}$ & $\svar{Tmd}{t_0}=\begin{cases}
						\attr{MD} & \text{if}\quad \attr{MD}\neq\undef\\
						t^-+n\attr{PRCL} & \text{else} \end{cases}$ \par
			with $n=ceil(\frac{\attr{NT}}{\attr{PRNXT}-\attr{NT}\yfr{t^-}{t^-+\attr{PRCL}}\attr{IPNR}})$ & where\par
						$t^- = \begin{cases} \attr{PRANX} & \text{if}\quad \attr{PRANX}\neq\undef \land \attr{PRANX}\geq t_0 \\
								\attr{IED}+\attr{PRCL} & \text{else if}\quad \attr{IED}+\attr{PRCL}\geq t_0 \\
								\sup t \in t^{PR}\mid t<t_0 & \text{else} \end{cases}$ \\
	\hline
  	$\svar{Nt}{}$ & & Same as PAM \\
	\hline
	$\svar{Ipnr}{}$ & & Same as PAM \\
  	\hline
  	$\svar{Ipac}{}$ & & Same as PAM \\
	\hline
  	$\svar{Fac}{}$ & & Same as PAM \\
  	\hline
  	$\svar{Nsc}{}$ & & Same as PAM \\
  	\hline
  	$\svar{Isc}{}$ & & Same as PAM \\
	\hline
  	$\svar{Prf}{}$ & & Same as PAM \\
	\hline
	$\svar{Sd}{}$ & & Same as PAM \\
	\hline
	$\svar{Prnxt}{}$ & $\svar{Prnxt}{t_0} = \sgn\attr{PRNXT}$ & \\
	\hline
	$\svar{Ipcb}{}$ & & Same as LAM \\
\end{states}


% ---------------------- table: nam functions ----------------------
\begin{functions}{NAM}
	AD & \pof{AD}{PAM} & \stf{AD}{PAM} \\
	\hline
	IED & \pof{IED}{PAM} & \stf{IED}{LAM} \\
	\hline
	PR & $\obs{rf}{\attr{CURS}}{t}\svar{Nsc}{t^-}(\svar{Prnxt}{t^-}-\svar{Ipac}{t^-}-$\par $\qquad\qquad \yfr{\svar{Sd}{t^-}}{t}\svar{Ipnr}{t^-}\svar{Ipcb}{t^-})$
		& {$\begin{aligned}
			\svar{Nt}{t^+} &= \svar{Nt}{t^-}-(\svar{Prnxt}{t^-}-\svar{Ipac}{t^+}) \\
			\svar{Ipac}{t^+} &= \svar{Ipac}{t^-} + \yfr{\svar{Sd}{t^-}}{t}\svar{Ipnr}{t^-}\svar{Ipcb}{t^-} \\
			\svar{Fac}{t^+} &= \begin{cases} \svar{Fac}{t^-} + \yfr{\svar{Sd}{t^-}}{t}\svar{Nt}{t^-}\attr{FER} & \text{if} \quad \attr{FEB}=\text{'N'} \\
					\frac{\yfr{t^{FP-}}{t}}{\yfr{t^{FP-}}{t^{FP+}}}\sgn\attr{FER} & \text{else} \end{cases} \\
			\svar{Ipcb}{t^+} &= \begin{cases} \svar{Ipcb}{t^-} & \text{if}\quad \attr{IPCB}\neq\text{'NT'} \\
							\svar{Nt}{t^+} & \text{else} \end{cases}\\
			\svar{Sd}{t^+} &= t \end{aligned}$} \par
	with\par
		{$\begin{aligned}
			t^{FP-} &= \sup t \in \vec{t}^{FP}\mid t<t_0 \\
			t^{FP+} &= \inf t \in \vec{t}^{FP}\mid t>t_0 \end{aligned}$} \\
	\hline
	PP & \pof{PP}{PAM}
		& \stf{PP}{LAM} \\
	\hline
	PY & \pof{PY}{PAM}
		& \stf{PY}{LAM} \\
	\hline
	FP & \pof{FP}{PAM}
		& \stf{FP}{LAM} \\
	\hline
  	PRD & \pof{PRD}{LAM}
		& \stf{PRD}{LAM} \\
	\hline
  	TD & \pof{TD}{LAM}
		& \stf{TD}{PAM} \\
	\hline
	IP & \pof{IP}{LAM}
		& \stf{IP}{PAM} \\
	\hline
	IPCI & \pof{IPCI}{PAM}
		& \stf{IPCI}{LAM} \\
	\hline
	IPCB & \pof{IPCB}{LAM}
		& \stf{IPCB}{LAM} \\
	\hline
	RR & \pof{RR}{PAM}
		& \stf{RR}{LAM} \\
	\hline
	RRF & \pof{RRF}{PAM}
		& \stf{RRF}{LAM} \\
	\hline
	SC & \pof{SC}{PAM}
		& \stf{SC}{LAM} \\
	\hline
	CE & \pof{CE}{PAM} & \stf{AD}{PAM} \\
\end{functions}



%%%%%%%%%%%%%%%%%%%% subsection: ann %%%%%%%%%%%%%%%%%%%%

\subsection{ANN: Annuity}\label{sec:ann}


% ---------------------- table: ann schedule ----------------------
\begin{schedule}{ANN}
	AD & & Same as PAM \\
	\hline
	IED & & Same as PAM \\
	\hline
	PR & & Same as LAM \\
	\hline
	PP & & Same as PAM \\
	\hline
	PY & & Same as PAM \\
	\hline
	FP & & Same as PAM \\
	\hline
	PRD & & Same as PAM \\
	\hline
	TD & & Same as PAM \\
	\hline
	IP & & Same as NAM \\
	\hline
	IPCI & & Same as PAM \\
  	\hline
	IPCB & & Same as LAM \\
	\hline
	RR & & Same as PAM \\
	\hline
	RRF & & Same as PAM \\
  	\hline
	SC & & Same as PAM \\
	\hline
	CE & & Same as PAM \\
\end{schedule}


% ---------------------- table: ann states ----------------------
\begin{states}{ANN}
	$\svar{Tmd}{}$ & & Same as NAM \\
	\hline
  	$\svar{Nt}{}$ & & Same as PAM \\
	\hline
	$\svar{Ipnr}{}$ & & Same as PAM \\
  	\hline
  	$\svar{Ipac}{}$ & & Same as PAM \\
	\hline
  	$\svar{Fac}{}$ & & Same as PAM \\
  	\hline
  	$\svar{Nsc}{}$ & & Same as PAM \\
  	\hline
  	$\svar{Isc}{}$ & & Same as PAM \\
	\hline
  	$\svar{Prf}{}$ & & Same as PAM \\
	\hline
	$\svar{Sd}{}$ & & Same as PAM \\
	\hline
	$\svar{Prnxt}{}$ & $\svar{Prnxt}{t_0} = \begin{cases} \sgn\attr{PRNXT} & \text{if}\quad \attr{PRNXT}\neq\undef \\
							(\attr{NT}+\svar{Ipac}{t_0})\frac{todo}{todo} & \text{else} \end{cases}$ & where $n=|\vec{t}|$ with $|a|$ indicating the cardinality of set $a$ \\
	\hline
	$\svar{Ipcb}{}$ & & Same as LAM \\
\end{states}


% ---------------------- table: ann functions ----------------------
\begin{functions}{ANN}
	AD & \pof{AD}{PAM} & \stf{AD}{PAM} \\
	\hline
	IED & \pof{IED}{PAM} & \stf{IED}{LAM} \\
	\hline
	PR & \pof{PR}{NAM} & \stf{PR}{NAM} \\
	\hline
	PP & \pof{PP}{PAM}
		& \stf{PP}{LAM} \\
	\hline
	PY & \pof{PY}{PAM}
		& \stf{PY}{LAM} \\
	\hline
	FP & \pof{FP}{PAM}
		& \stf{FP}{LAM} \\
	\hline
  	PRD & \pof{PRD}{LAM}
		& \stf{PRD}{LAM} \\
	\hline
  	TD & \pof{TD}{LAM}
		& \stf{TD}{PAM} \\
	\hline
	IP & \pof{IP}{LAM}
		& \stf{IP}{PAM} \\
	\hline
	IPCI & \pof{IPCI}{PAM}
		& \stf{IPCI}{LAM} \\
	\hline
	IPCB & \pof{IPCB}{LAM}
		& \stf{IPCB}{LAM} \\
	\hline
	RR & \pof{RR}{PAM}
		& {$\begin{aligned}
			\svar{Ipac}{t^+} &= \svar{Ipac}{t^-} + \yfr{\svar{Sd}{t^-}}{t}\svar{Ipnr}{t^-}\svar{Ipcb}{t^-} \\
			\svar{Fac}{t^+} &= \begin{cases} \svar{Fac}{t^-} + \yfr{\svar{Sd}{t^-}}{t}\svar{Nt}{t^-}\attr{FER} & \text{if} \quad \attr{FEB}=\text{'N'} \\
					\frac{\yfr{t^{FP-}}{t}}{\yfr{t^{FP-}}{t^{FP+}}}\sgn\attr{FER} & \text{else} \end{cases} \\
			\svar{Ipnr}{t^+} &= \min(\max(\svar{Ipnr}{t^-}+\Delta r,\attr{RRLF}),\attr{RRLC}) \\
			\svar{Prnxt}{t^+} &= \ann{t}{\svar{Tmd}{t^+}}{\svar{Nt}{t^+}}{\svar{Ipac}{t^+}}{\svar{Ipnr}{t^+}} \\
			\svar{Sd}{t^+} &= t \end{aligned}$}\par
		with\par
		{$\begin{aligned}
			\Delta r &= \min(\max(\obs{rf}{\attr{RRMO}}{t}\attr{RRMT}+\attr{RRSP} - \svar{Ipnr}{t^-},\attr{RRPF}),\attr{RRPC}) \\
			t^{FP-} &= \sup t \in \vec{t}^{FP}\mid t<t_0 \\
			t^{FP+} &= \inf t \in \vec{t}^{FP}\mid t>t_0 \end{aligned}$} \\
  	\hline
	RRF & \pof{RRF}{PAM}
		& {$\begin{aligned}
			\svar{Ipac}{t^+} &= \svar{Ipac}{t^-} + \yfr{\svar{Sd}{t^-}}{t}\svar{Ipnr}{t^-}\svar{Ipcb}{t^-} \\
			\svar{Fac}{t^+} &= \begin{cases} \svar{Fac}{t^-} + \yfr{\svar{Sd}{t^-}}{t}\svar{Nt}{t^-}\attr{FER} & \text{if} \quad \attr{FEB}=\text{'N'} \\
					\frac{\yfr{t^{FP-}}{t}}{\yfr{t^{FP-}}{t^{FP+}}}\sgn\attr{FER} & \text{else} \end{cases} \\
			\svar{Ipnr}{t^+} &= \attr{RRNXT} \\
			\svar{Prnxt}{t^+} &= \ann{t}{\svar{Tmd}{t^+}}{\svar{Nt}{t^+}}{\svar{Ipac}{t^+}}{\svar{Ipnr}{t^+}} \\
			\svar{Sd}{t^+} &= t \end{aligned}$} \par
		with\par
		{$\begin{aligned}
			t^{FP-} &= \sup t \in \vec{t}^{FP}\mid t<t_0 \\
			t^{FP+} &= \inf t \in \vec{t}^{FP}\mid t>t_0 \end{aligned}$} \\
	\hline
	SC & \pof{SC}{PAM}
		& \stf{SC}{LAM} \\
	\hline
	CE & \pof{CE}{PAM} & \stf{AD}{PAM} \\
\end{functions}



%%%%%%%%%%%%%%%%%%%% subsection: clm %%%%%%%%%%%%%%%%%%%%

\subsection{CLM: Call Money}\label{sec:clm}


% ---------------------- table: clm schedule ----------------------
\begin{schedule}{CLM}
	AD & & Same as PAM \\
	\hline
	IED & & Same as PAM \\
	\hline
	PR & & Same as PAM \\
	\hline
	FP & & Same as PAM \\
	\hline
	IP & $t^{IP}=\svar{Tmd}{t_0}$ & \\
	\hline
	IPCI & $\vec{t}^{IPCI} = \begin{cases}
					\undef & \text{if}\quad \attr{IPNR}=\undef \\
					\sdl{s}{\attr{IPCL}}{\svar{Tmd}{t_0}} & \text{else} \end{cases}$ &
				where\par
				$s=\begin{cases} \attr{IPANX} & \text{if}\quad \attr{IPANX}\neq\undef\\
						\attr{IED}+\attr{IPCL} & \text{else} \end{cases}$ \\
	\hline
	RR & & Same as PAM \\
	\hline
	RRF & & Same as PAM \\
	\hline
	CE & & Same as PAM \\
\end{schedule}


% ---------------------- table: clm states ----------------------
\begin{states}{CLM}
	$\svar{Tmd}{}$ & $\svar{Tmd}{t_0} = \begin{cases} \attr{MD} & \text{if}\quad \attr{MD}\neq\undef \\
							s & \text{else if}\quad \obs{ev}{\attr{CID}}{t_0}\neq\{\} \\
							\tmax & \text{else} \end{cases}$
			& where\par
			$s=\sup t\in\tev{\obs{ev}{\attr{CID}}{t_0}}$ \\
	\hline
  	$\svar{Nt}{}$ & & Same as PAM \\
	\hline
	$\svar{Ipnr}{}$ & & Same as PAM \\
  	\hline
  	$\svar{Ipac}{}$ & & Same as PAM \\
	\hline
  	$\svar{Fac}{}$ & & Same as PAM \\
  	\hline
  	$\svar{Prf}{}$ & & Same as PAM \\
	\hline
	$\svar{Sd}{}$ & & Same as PAM \\
\end{states}


% ---------------------- table: clm functions ----------------------
\begin{functions}{CLM}
	AD & \pof{AD}{PAM} & \stf{AD}{PAM} \\
	\hline
	IED & $\obs{rf}{\attr{CURS}}{t}\sgn (-1)\attr{NT}$ & \stf{IED}{PAM} \\
	\hline
	PR & \pof{PR}{PAM} & \stf{PR}{PAM} \\
	\hline
	FP & \pof{FP}{PAM}
		& \stf{FP}{PAM} \\
	\hline
	IP & $\obs{rf}{\attr{CURS}}{t}(\svar{Ipac}{t^-}+\yfr{\svar{Sd}{t^-}}{t}\svar{Ipnr}{t^-}\svar{Nt}{t^-})$
		& {$\begin{aligned}
				\svar{Ipac}{t^+} &= 0.0 \\
				\svar{Sd}{t^+} &= t \end{aligned}$} \\
	\hline
	IPCI & \pof{IPCI}{PAM}
		& \stf{IPCI}{PAM} \\
	\hline
	RR & \pof{RR}{PAM}
		& \stf{RR}{PAM} \\
	\hline
	RRF & \pof{RRF}{PAM}
		& \stf{RRF}{PAM} \\
	\hline
	CE & \pof{CE}{PAM} & \stf{AD}{PAM} \\
\end{functions}



%%%%%%%%%%%%%%%%%%%% subsection: ump %%%%%%%%%%%%%%%%%%%%

\subsection{UMP: Undefined Maturity Profile}\label{sec:ump}


% ---------------------- table: ump schedule ----------------------
\begin{schedule}{UMP}
	AD & & Same as PAM \\
	\hline
	IED & & Same as PAM \\
	\hline
	PR & $\vec{t}^{PR}=\obs{ev}{\attr{CID},i}{t_0}$ & with $i\in\{\attr{PR},\attr{PI}\}$\\
	\hline
	FP & & Same as PAM \\
	\hline
	IPCI & $\vec{t}^{IPCI} = \begin{cases}
					\undef & \text{if}\quad \attr{IPNR}=\undef \\
					\sdl{s}{\attr{IPCL}}{\svar{Tmd}{t_0}} & \text{else} \end{cases}$ &
				where\par
				$s=\begin{cases} \attr{IPANX} & \text{if}\quad \attr{IPANX}\neq\undef\\
						\attr{IED}+\attr{IPCL} & \text{else} \end{cases}$ \\
	\hline
	RR & & Same as PAM \\
	\hline
	RRF & & Same as PAM \\
	\hline
	CE & & Same as PAM \\
\end{schedule}


% ---------------------- table: ump states ----------------------
\begin{states}{UMP}
	$\svar{Tmd}{}$ & $\svar{Tmd}{t_0} = \begin{cases} s & \text{if}\quad \obs{ev}{\attr{CID}}{t_0}\neq\{\} \\
							\tmax & \text{else} \end{cases}$
			& where\par
			$s=\sup t, t\in\tev{\obs{ev}{\attr{CID},i}{t_0}}$\par
			$i\in\{\attr{PR},\attr{PI}\}$ \\
	\hline
  	$\svar{Nt}{}$ & & Same as PAM \\
	\hline
	$\svar{Ipnr}{}$ & & Same as PAM \\
  	\hline
  	$\svar{Ipac}{}$ & & Same as PAM \\
	\hline
  	$\svar{Fac}{}$ & & Same as PAM \\
  	\hline
  	$\svar{Prf}{}$ & & Same as PAM \\
	\hline
	$\svar{Sd}{}$ & & Same as PAM \\
\end{states}


% ---------------------- table: ump functions ----------------------
\begin{functions}{UMP}
	AD & \pof{AD}{PAM} & \stf{AD}{PAM} \\
	\hline
	IED & \pof{IED}{CLM} & \stf{IED}{PAM} \\
	\hline
	PR & $\fev{e_t^{PR}}$ & {$\begin{aligned}
				\svar{Ipac}{t^+} &= \svar{Ipac}{t^-}+\yfr{\svar{Sd}{t^-}}{t}\svar{Ipnr}{t^-}\svar{Nt}{t^-} \\
				\svar{Fac}{t^+} &= \begin{cases} \svar{Fac}{t^-} + \yfr{\svar{Sd}{t^-}}{t}\svar{Nt}{t^-}\attr{FER} & \text{if} \quad \attr{FEB}=\text{'N'} \\
					\frac{\yfr{t^-}{t}}{\yfr{t^-}{t^+}}\attr{FER} & \text{else} \end{cases} \\
				\svar{Nt}{t^+} &= \svar{Nt}{t^-}-\fev{e_t^{PR}} \\
				\svar{Sd}{t^+} &= t \end{aligned}$} \\
	\hline
	FP & \pof{FP}{PAM}
		& \stf{FP}{PAM} \\
	\hline
	IPCI & \pof{IPCI}{PAM}
		& \stf{IPCI}{PAM} \\
	\hline
	RR & \pof{RR}{PAM}
		& \stf{RR}{PAM} \\
	\hline
	RRF & \pof{RRF}{PAM}
		& \stf{RRF}{PAM} \\
	\hline
	CE & \pof{CE}{PAM} & \stf{AD}{PAM} \\
\end{functions}



%%%%%%%%%%%%%%%%%%%% subsection: csh %%%%%%%%%%%%%%%%%%%%

\subsection{CSH: Cash}\label{sec:csh}


% ---------------------- table: csh schedule ----------------------
\begin{schedule}{CSH}
	AD & & Same as PAM \\
\end{schedule}


% ---------------------- table: csh states ----------------------
\begin{states}{CSH}
  	$\svar{Nt}{}$ & $\svar{Nt}{t_0}=\sgn\attr{NT}$ & \\
	\hline
	$\svar{Sd}{}$ & & Same as PAM \\
\end{states}


% ---------------------- table: csh functions ----------------------
\begin{functions}{CSH}
	AD & \pof{AD}{PAM} & $\svar{Sd}{t^+}=t$ \\
\end{functions}



%%%%%%%%%%%%%%%%%%%% subsection: stk %%%%%%%%%%%%%%%%%%%%

\subsection{STK: Stock}\label{sec:stk}


% ---------------------- table: stk schedule ----------------------
\begin{schedule}{STK}
	AD & & Same as PAM \\
	\hline
	PRD & & Same as PAM \\
	\hline
	TD & & Same as PAM \\
	\hline
	DV$^{(fix)}$ & $t^{DV^{(fix)}} = \begin{cases}
					\undef & \text{if}\quad \attr{DVNP}=\undef \\
					\attr{DVANX} & \text{else} \end{cases}$ & \\
	\hline
	DV & $\vec{t}^{DV} = \begin{cases}
					\undef & \text{if}\quad \attr{DVANX}=\undef \land \attr{DVCL}=\undef \\
					\sdl{s}{\attr{DVCL}}{\tmax} & \text{else} \end{cases}$
		& where\par
		$s=\begin{cases} \attr{DVANX} & \text{if}\quad \attr{DVNP}=\undef \\
				\attr{DVANX}+\attr{DVCL} & \text{else} \end{cases}$ \\
	\hline
	CE & & Same as PAM \\
\end{schedule}


% ---------------------- table: stk states ----------------------
\begin{states}{STK}
  	$\svar{Prf}{}$ & & Same as PAM \\
	\hline
	$\svar{Sd}{}$ & & Same as PAM \\
\end{states}


% ---------------------- table: stk functions ----------------------
\begin{functions}{STK}
	AD & \pof{AD}{PAM} & $\svar{Sd}{t^+}=t$ \\
	\hline
	PRD & $\obs{rf}{\attr{CURS}}{t}\sgn (-1) \attr{PPRD}$
		& $\svar{Sd}{t^+}=t$ \\
	\hline
	TD & $\obs{rf}{\attr{CURS}}{t}\sgn \attr{PTD}$
		& $\svar{Sd}{t^+}=t$ \\
	\hline
	DV$^{(fix)}$ & $\obs{rf}{\attr{CURS}}{t}\sgn\attr{DVNP}$ & $\svar{Sd}{t^+}=t$ \\
	\hline
	DV & $\obs{rf}{\attr{CURS}}{t}\sgn \obs{rf}{\attr{DVMO}}{t}$
		& $\svar{Sd}{t^+}=t$ \\
	\hline
	CE & \pof{CE}{PAM} & \stf{AD}{STK} \\
\end{functions}



%%%%%%%%%%%%%%%%%%%% subsection: com %%%%%%%%%%%%%%%%%%%%

\subsection{COM: Commodity}\label{sec:com}


% ---------------------- table: com schedule ----------------------
\begin{schedule}{COM}
	AD & & Same as PAM \\
	\hline
	PRD & & Same as STK \\
	\hline
	TD & & Same as STK \\
\end{schedule}


% ---------------------- table: com states ----------------------
\begin{states}{COM}
	$\svar{Sd}{}$ & & Same as STK \\
\end{states}


% ---------------------- table: com functions ----------------------
\begin{functions}{COM}
	AD & \pof{AD}{PAM} & \stf{AD}{STK} \\
	\hline
	PRD & \pof{PRD}{STK} & \stf{PRD}{STK} \\
	\hline
	TD & \pof{PRD}{STK} & \stf{PRD}{STK} \\
\end{functions}



%%%%%%%%%%%%%%%%%%%% subsection: fxout %%%%%%%%%%%%%%%%%%%%

\subsection{FXOUT: Foreign Exchange Outright}\label{sec:fxout}


% ---------------------- table: fxout schedule ----------------------
\begin{schedule}{FXOUT}
	AD & & Same as PAM \\
	\hline
	PRD & & Same as PAM \\
	\hline
	TD & & Same as PAM \\
	\hline
	STD & $t^{STD} = \begin{cases}
					\undef & \text{if}\quad \attr{DS}=\text{'D'} \\
					\svar{Tmd}{t_0} & \text{else} \end{cases}$ & \\
	\hline
	STD$^{(1)}$ & $t^{STD} = \begin{cases}
					\undef & \text{if}\quad \attr{DS}=\text{'S'} \\
					\svar{Tmd}{t_0} & \text{else} \end{cases}$ & \\
	\hline
	STD$^{(2)}$ & $t^{STD} = \begin{cases}
					\undef & \text{if}\quad \attr{DS}=\text{'S'} \\
					\svar{Tmd}{t_0} & \text{else} \end{cases}$ & \\
	\hline
	CE & & Same as PAM \\
\end{schedule}


% ---------------------- table: fxout states ----------------------
\begin{states}{FXOUT}
	$\svar{Tmd}{}$ & $\svar{Tmd}{t_0} = \begin{cases} \attr{MD} & \text{if}\quad \attr{STD}=\undef\\
							\attr{STD} & \text{else} \end{cases}$ & \\
	\hline
  	$\svar{Prf}{}$ & & Same as PAM \\
	\hline
	$\svar{Sd}{}$ & & Same as PAM \\
\end{states}


% ---------------------- table: fxout functions ----------------------
\begin{functions}{FXOUT}
	AD & \pof{AD}{PAM} & \stf{AD}{STK} \\
	\hline
	PRD & \pof{PRD}{STK} & \stf{PRD}{STK} \\
	\hline
	TD & \pof{TD}{STK} & \stf{TD}{STK} \\
	\hline
	STD & $\obs{rf}{\attr{CURS}}{t}\sgn (\attr{NT}-\obs{rf}{i}{\svar{Tmd}{t}}\attr{NT2})$\par
		where\par
		$i=concat(\text{\attr{CUR2},"/",\attr{CUR}})$\par
		and $concat(x,y,z)$ indicates the string concatenation function
		& $\svar{Sd}{t^+}=t$ \\
	\hline
	STD$^{(1)}$ & $\obs{rf}{\attr{CURS}}{t}\sgn\attr{NT}$ & $\svar{Sd}{t^+}=t$ \\
	\hline
	STD$^{(2)}$ & $\obs{rf}{\attr{CURS}}{t}\sgn (-1)\attr{NT2}$ & $\svar{Sd}{t^+}=t$ \\
	\hline
	CE & \pof{CE}{PAM} & \stf{CD}{STK} \\
\end{functions}



%%%%%%%%%%%%%%%%%%%% subsection: swppv %%%%%%%%%%%%%%%%%%%%

\subsection{SWPPV: Plain Vanilla Interest Rate Swap}\label{sec:swppv}


% ---------------------- table: swppv schedule ----------------------
\begin{schedule}{SWPPV}
	AD & & Same as PAM \\
	\hline
	PRD & & Same as PAM \\
	\hline
	TD & & Same as PAM \\
	\hline
	IED & & Same as PAM \\
	\hline
	PR & & Same as PAM \\
	\hline
	IP & $\vec{t}^{IP} = \begin{cases}
					\undef & \text{if}\quad \attr{DS}=\text{'D'} \\
					\svar{Tmd}{t_0} & \text{else if} \attr{IPCL}=\undef \\
					\sdl{s}{\attr{IPCL}}{\svar{Tmd}{t_0}} & \text{else} \end{cases}$
		& where\par
			$s=\begin{cases} \attr{IPANX} & \text{if}\quad \attr{IPANX}\neq\undef \\
					\attr{IED}+\attr{IPCL} & \text{else} \end{cases}$\\
	\hline
	IP$^{(fix)}$ & $\vec{t}^{IP^{(fix)}} = \begin{cases}
					\undef & \text{if}\quad \attr{DS}=\text{'S'} \\
					\svar{Tmd}{t_0} & \text{else if} \attr{IPCL}=\undef \\
					\sdl{s}{\attr{IPCL}}{\svar{Tmd}{t_0}} & \text{else} \end{cases}$ & \\
	\hline
	IP$^{(var)}$ & $\vec{t}^{IP^{(var)}} = \begin{cases}
					\undef & \text{if}\quad \attr{DS}=\text{'S'} \\
					\svar{Tmd}{t_0} & \text{else if} \attr{IPCL}=\undef \\
					\sdl{s}{\attr{IPCL}}{\svar{Tmd}{t_0}} & \text{else} \end{cases}$ & \\
	\hline
	RR & $\vec{t}^{RR}=\sdl{s}{\attr{RRCL}}{\svar{Tmd}{t_0}}$
			& where\par
				$s=\begin{cases} \attr{RRANX} & \text{if}\quad \attr{RRANX}\neq\undef \\
						\attr{IED}+\attr{RRCL} & \text{else} \end{cases}$ \\
	\hline
	CE & & Same as PAM \\
\end{schedule}


% ---------------------- table: swppv states ----------------------
\begin{states}{SWPPV}
	$\svar{Tmd}{}$ &  & Same as PAM \\
	\hline
	$\svar{Nt}{}$ &  & Same as PAM \\
	\hline
	$\svar{Ipnr}{}$ & $\svar{Ipnr}{t_0} = \begin{cases} 0.0 & \text{if} \quad \attr{IED} > t_0 \\
							\attr{IPNR2} & \text{else} \end{cases}$ & \\
  	\hline
  	$\svar{Ipac}{}$ & $\svar{Ipac}{t_0} = \begin{cases} \attr{IPAC} & \text{if} \quad \attr{IPAC} \neq \undef \\
						\yfr{t^-}{t_0}\svar{Nt}{t_0}(\attr{IPNR}-\svar{Ipnr}{t_0}) & \text{else} \end{cases}$ &
			with $t^- = \sup{t}, t \in t^{IP}, t<t_0$ \\
	\hline
	$\svar{Ipac1}{}$ & $\svar{Ipac1}{t_0} = \yfr{t^-}{t_0}\svar{Nt}{t_0}\attr{IPNR}$ & with $t^- = \sup{t}, t \in t^{IP}, t<t_0$ \\
	\hline
	$\svar{Ipac2}{}$ & $\svar{Ipac2}{t_0} = \yfr{t^-}{t_0}\svar{Nt}{t_0}\svar{Ipnr}{t_0}$ & with $t^- = \sup{t}, t \in t^{IP}, t<t_0$ \\
	\hline
  	$\svar{Prf}{}$ & & Same as PAM \\
	\hline
	$\svar{Sd}{}$ & & Same as PAM \\
\end{states}


% ---------------------- table: swppv functions ----------------------
\begin{functions}{SWPPV}
	AD & \pof{AD}{PAM} & {$\begin{aligned}
				\svar{Ipac}{t^+} &= \yfr{\svar{Sd}{t^-}}{t}\svar{Nt}{t_0}(\attr{IPNR}-\svar{Ipnr}{t_0}) \\
				\svar{Ipac1}{t^+} &= \yfr{\svar{Sd}{t^-}}{t}\svar{Nt}{t_0}\attr{IPNR} \\
				\svar{Ipac2}{t^+} &= \yfr{\svar{Sd}{t^-}}{t}\svar{Nt}{t_0}\svar{Ipnr}{t_0} \\
				\svar{Sd}{t^+} &= t \end{aligned}$} \\
	\hline
	IED & 0.0 & {$\begin{aligned}
				\svar{Nt}{t^+} &= \sgn\attr{NT}\\
				\svar{Ipac}{t^+} &= 0.0 \\
				\svar{Ipac1}{t^+} &= 0.0 \\
				\svar{Ipac2}{t^+} &= 0.0 \\
				\svar{Ipnr}{t^+} &= \attr{IPNR2} \\
				\svar{Sd}{t^+} &= t \end{aligned}$} \\
	\hline
	PR & 0.0 & {$\begin{aligned}
				\svar{Nt}{t^+} &= 0.0\\
				\svar{Ipnr}{t^+} &= 0.0 \\
				\svar{Sd}{t^+} &= t \end{aligned}$} \\
	\hline
	PRD & \pof{PRD}{STK} & {$\begin{aligned}
				\svar{Ipac}{t^+} &= \yfr{\svar{Sd}{t^-}}{t}\svar{Nt}{t_0}(\attr{IPNR}-\svar{Ipnr}{t_0}) \\
				\svar{Ipac1}{t^+} &= \yfr{\svar{Sd}{t^-}}{t}\svar{Nt}{t_0}\attr{IPNR} \\
				\svar{Ipac2}{t^+} &= \yfr{\svar{Sd}{t^-}}{t}\svar{Nt}{t_0}\svar{Ipnr}{t_0} \\
				\svar{Sd}{t^+} &= t \end{aligned}$} \\
	\hline
	TD & \pof{TD}{STK} & {$\begin{aligned}
				\svar{Nt}{t^+} &= 0.0\\
				\svar{Ipac}{t^+} &= 0.0 \\
				\svar{Ipac1}{t^+} &= 0.0 \\
				\svar{Ipac2}{t^+} &= 0.0 \\
				\svar{Ipnr}{t^+} &= 0.0 \\
				\svar{Sd}{t^+} &= t \end{aligned}$} \\
	\hline
	IP & $\obs{rf}{\attr{CURS}}{t}\sgn (\svar{Ipac}{t^-}+\yfr{\svar{Sd}{t^-}}{t}(\attr{IPNR}-\svar{Ipnr}{t^-})\svar{Nt}{t^-})$ & {$\begin{aligned}
				\svar{Ipac}{t^+} &= 0.0 \\
				\svar{Sd}{t^+} &= t \end{aligned}$} \\
	\hline
	IP$^{(fix})$ & $\obs{rf}{\attr{CURS}}{t}\sgn (\svar{Ipac1}{t^-}+\yfr{\svar{Sd}{t^-}}{t}\attr{IPNR}\svar{Nt}{t^-})$ & {$\begin{aligned}
				\svar{Ipac1}{t^+} &= 0.0 \\
				\svar{Sd}{t^+} &= t \end{aligned}$} \\
	\hline
	IP$^{(var)}$ & $\obs{rf}{\attr{CURS}}{t}\sgn (\svar{Ipac2}{t^-}-\yfr{\svar{Sd}{t^-}}{t}\svar{Ipnr}{t^-}\svar{Nt}{t^-})$ & {$\begin{aligned}
				\svar{Ipac2}{t^+} &= 0.0 \\
				\svar{Sd}{t^+} &= t \end{aligned}$} \\
	\hline
	RR & \pof{RR}{PAM} & {$\begin{aligned}
				\svar{Ipac}{t^+} &= \yfr{\svar{Sd}{t^-}}{t}\svar{Nt}{t_0}(\attr{IPNR}-\svar{Ipnr}{t_0}) \\
				\svar{Ipac1}{t^+} &= \yfr{\svar{Sd}{t^-}}{t}\svar{Nt}{t_0}\attr{IPNR} \\
				\svar{Ipac2}{t^+} &= \yfr{\svar{Sd}{t^-}}{t}\svar{Nt}{t_0}\svar{Ipnr}{t_0} \\
				\svar{Ipnr}{t^+} &= \attr{RRMLT}\obs{rf}{\attr{RRMO}}{t}+\attr{RRSP} \\
				\svar{Sd}{t^+} &= t \end{aligned}$} \\
\hline
	CE & \pof{CE}{PAM} & \stf{AD}{SWPPV} \\
\end{functions}



%%%%%%%%%%%%%%%%%%%% subsection: swaps %%%%%%%%%%%%%%%%%%%%

\subsection{SWAPS: Swap}\label{sec:swaps}


% ---------------------- table: swaps schedule ----------------------
\begin{schedule}{SWAPS}
	AD & & Same as PAM \\
	\hline
	PRD & & Same as PAM \\
	\hline
	TD & & Same as PAM \\
	\hline
	$k$ & $\{e_t^k\} = \begin{cases}
				\{e_t^{k,1}\}\cup \{e_s^{l,2}\} & \text{if}\quad \attr{DS}=\text{'D'} \\
				\{e_t^{k,1}\} + \{e_s^{l,2}\} & \text{else} \end{cases}$ \par
	with\par
	$\{e_t^{k,1}\}+\{e_s^{l,2}\} = U \cup V$\par
	and\par
	{$\begin{aligned}
		U &= \{e_t^{k,1}\} \triangle \{e_s^{l,2}\}\\
		V &= \{x_\tau^m+y_\tau^m\}
	\end{aligned}$} \par
	where for any two events $x_t^k\in\{e_t^{k,1}\}$, $y_s^l\in\{e_s^{l,2}\}$ we have $x_t^k=y_s^l \iff t=s \land k=l$, $\triangle$ is the \textit{distinct union}-operator, and $x_t^k+y_s^l=z_\tau^m$ with $\tau=t=s$, $m=k=l\in\{\text{IED,IP,PR}\}$ indicates that any two congruent events of type IED, IP, or PR are \textit{merged} into a new \textit{aggregate} event (see payoff and state transition function below).
		& with\par
			{$\begin{aligned}
				\{e_t^{k,1}\} &= \cldev{\attr{CTST}_{FirstLeg}^{Contract}}{t_0}{\attr{CNTRL}=r^{(1)}} \\
				\{e_s^{l,2}\} &= \cldev{\attr{CTST}_{SecondLeg}^{Contract}}{t_0}{\attr{CNTRL}=r^{(2)}} \\
				r^{(1)} &= \begin{cases} RPA & \text{if}\quad \attr{CNTRL}=RFL \\
					RPL & \text{else} \end{cases} \\
				r^{(2)} &= \begin{cases} RPL & \text{if}\quad \attr{CNTRL}=RFL \\
					RPA & \text{else} \end{cases}
			\end{aligned}$} \\
	\hline
	CE & & Same as PAM \\
\end{schedule}


% ---------------------- table: swaps states ----------------------
\begin{states}{SWAPS}
	$\svar{Tmd}{}$ & $\svar{Tmd}{t_0}=\max(\cldsvs{\attr{CTST}_{FirstLeg}^{Contract}}{t_0}{Tmd}, \cldsvs{\attr{CTST}_{SecondLeg}^{Contract}}{t_0}{Tmd})$ & \\
	\hline
  	$\svar{Ipac}{}$ & $\svar{Ipac}{t_0} = \cldsv{\attr{CTST}_{FirstLeg}^{Contract}}{t_0}{Ipac}{\attr{CNTRL}=r^{(1)}} + \cldsv{\attr{CTST}_{SecondLeg}^{Contract}}{t_0}{Ipac}{\attr{CNTRL}=r^{(2)}}$
			& with\par
			{$\begin{aligned}
				r^{(1)} &= \begin{cases} RPA & \text{if}\quad \attr{CNTRL}=RFL \\
					RPL & \text{else} \end{cases} \\
				r^{(2)} &= \begin{cases} RPL & \text{if}\quad \attr{CNTRL}=RFL \\
					RPA & \text{else} \end{cases}
			\end{aligned}$} \\
	\hline
  	$\svar{Prf}{}$ & & Same as PAM \\
	\hline
	$\svar{Sd}{}$ & & Same as PAM \\
\end{states}


% ---------------------- table: swaps functions ----------------------
\begin{functions}{SWAPS}
	AD & \pof{AD}{PAM} & {$\begin{aligned}
				\svar{Ipac}{t^+} &= \cldsv{\attr{CTST}_{FirstLeg}^{Contract}}{t}{Ipac}{\attr{CNTRL}=r^{(1)}} \\
						&+ \cldsv{\attr{CTST}_{SecondLeg}^{Contract}}{t}{Ipac}{\attr{CNTRL}=r^{(2)}} \\
				\svar{Fac}{t^+} &= \begin{cases} \svar{Fac}{t^-} + \yfr{\svar{Sd}{t^-}}{t}\svar{Nt}{t^-}\attr{FER} & \text{if} \quad \attr{FEB}=\text{'N'} \\
					\frac{\yfr{t^-}{t}}{\yfr{t^-}{t^+}}\attr{FER} & \text{else} \end{cases} \\
				\svar{Sd}{t^+} &= t \end{aligned}$} \\
	\hline
	PRD & $\obs{rf}{\attr{CURS}}{t} ( (-1)\attr{PPRD}$ \par
				 $\qquad + \cldsv{\attr{CTST}_{FirstLeg}^{Contract}}{t}{Ipac}{\attr{CNTRL}=r^{(1)}}$ \par
				 $\qquad + \cldsv{\attr{CTST}_{SecondLeg}^{Contract}}{t}{Ipac}{\attr{CNTRL}=r^{(2)}})$
		& {$\begin{aligned}
				\svar{Ipac}{t^+} &= \cldsv{\attr{CTST}_{FirstLeg}^{Contract}}{t}{Ipac}{\attr{CNTRL}=r^{(1)}} \\
						&+ \cldsv{\attr{CTST}_{SecondLeg}^{Contract}}{t}{Ipac}{\attr{CNTRL}=r^{(2)}} \\
				\svar{Sd}{t^+} &= t \end{aligned}$} \\
	\hline
	TD & $\obs{rf}{\attr{CURS}}{t} (\attr{PTD}$ \par
				 $\qquad + \cldsv{\attr{CTST}_{FirstLeg}^{Contract}}{t}{Ipac}{\attr{CNTRL}=r^{(1)}}$ \par
				 $\qquad + \cldsv{\attr{CTST}_{SecondLeg}^{Contract}}{t}{Ipac}{\attr{CNTRL}=r^{(2)}})$
		& {$\begin{aligned}
				\svar{Ipac}{t^+} &= 0.0 \\
				\svar{Sd}{t^+} &= t \end{aligned}$} \\
	\hline
	$z_\tau^m$ & $\fev{x_\tau^m}+\fev{y_\tau^m}$
		& {$\begin{aligned}
				\svar{Ipac}{t^+} &= \cldsv{\attr{CTST}_{FirstLeg}^{Contract}}{t}{Ipac}{\attr{CNTRL}=r^{(1)}} \\						&+ \cldsv{SecondLeg}{t}{Ipac}{\attr{CNTRL}=r^{(2)}} \\
				\svar{Sd}{t^+} &= t \end{aligned}$} \\
	\hline
	CE & \pof{CE}{PAM} & \stf{AD}{SWAPS} \\
\end{functions}


%%%%%%%%%%%%%%%%%%%% subsection: capfl %%%%%%%%%%%%%%%%%%%%

\subsection{CAPFL: Cap-Floor}\label{sec:capfl}


% ---------------------- table: capfl schedule ----------------------
\begin{schedule}{CAPFL}
	AD & & Same as PAM \\
	\hline
	PRD & & Same as PAM \\
	\hline
	TD & & Same as PAM \\
	\hline
	$k$ & $\{e_t^k\} = \{x_t^k+y_s^l\}$ \par
	for all events $x_t^k\in\cldev{\attr{CTST}_{Underlying}^{Contract}}{t_0}{\attr{CNTRL}=RPA}$, $y_s^l\in\cldev{\attr{CTST}_{Underlying}^{Contract}}{t_0}{\attr{CNTRL}=RPA,\attr{RRLC}=\attr{RRLC},\attr{RRLF}=\attr{RRLF}}$ with $t=s \land k=l=IP$. That is, any two congruent events of the child-contract schedule, evaluated once without $\attr{RRLC},\attr{RRLF}$ defined and once with the attributes defined, which are of type IP are \textit{merged} into a new \textit{aggregate} event (see payoff and state transition function below). & \\
	\hline
	CE & & Same as PAM \\
\end{schedule}


% ---------------------- table: capfl states ----------------------
\begin{states}{CAPFL}
	$\svar{Tmd}{}$ & $\svar{Tmd}{t_0}=\max(\cldsv{\attr{CTST}_{Underlying}^{Contract}}{t_0}{Tmd}{\attr{CNTRL}=RPA},$ \par
			$\qquad \cldsv{\attr{CTST}_{Underlying}^{Contract}}{t_0}{Tmd}{\attr{CNTRL}=RPA,\attr{RRLC}=\attr{RRLC},\attr{RRLF}=\attr{RRLF}})$ & \\
	\hline
  	$\svar{Ipac}{}$ & $\svar{Ipac}{t_0} = \sgn abs($ \par
			$\qquad\cldsv{\attr{CTST}_{Underlying}^{Contract}}{t_0}{Ipac}{\attr{CNTRL}=r^{(1)}}$\par
			$\qquad - \cldsv{\attr{CTST}_{Underlying}^{Contract}}{t_0}{Ipac}{\attr{CNTRL}=RPA,\attr{RRLC}=\attr{RRLC},\attr{RRLF}=\attr{RRLF}})$
			& \\
	\hline
  	$\svar{Prf}{}$ & & Same as PAM \\
	\hline
	$\svar{Sd}{}$ & & Same as PAM \\
\end{states}


% ---------------------- table: capfl functions ----------------------
\begin{functions}{CAPFL}
	AD & \pof{AD}{PAM} & {$\begin{aligned}
				\svar{Ipac}{t^+} &= \sgn  abs(\cldsv{\attr{CTST}_{Underlying}^{Contract}}{t}{Ipac}{\attr{CNTRL}=RPA} \\
						&- \cldsv{\attr{CTST}_{Underlying}^{Contract}}{t}{Ipac}{\attr{CNTRL}=RPA, \\
						&\qquad \attr{RRLC}=\attr{RRLC},\attr{RRLF}=\attr{RRLF}}) \\
				\svar{Sd}{t^+} &= t \end{aligned}$} \\
	\hline
	PRD & $\obs{rf}{\attr{CURS}}{t} ( (-1)\attr{PPRD} + \sgn  abs($ \par
				 $\qquad \cldsv{\attr{CTST}_{Underlying}^{Contract}}{t}{Ipac}{\attr{CNTRL}=RPA} - $ \par
				 $\qquad \cldsv{\attr{CTST}_{Underlying}^{Contract}}{t}{Ipac}{\attr{CNTRL}=RPA,$ \par
					$\qquad\qquad\attr{RRLC}=\attr{RRLC},\attr{RRLF}=\attr{RRLF}}))$
		& {$\begin{aligned}
				\svar{Ipac}{t^+} &= \sgn abs(\cldsv{\attr{CTST}_{Underlying}^{Contract}}{t}{Ipac}{\attr{CNTRL}=RPA} \\
						&- \cldsv{\attr{CTST}_{Underlying}^{Contract}}{t}{Ipac}{\attr{CNTRL}=RPA, \\
						&\qquad \attr{RRLC}=\attr{RRLC},\attr{RRLF}=\attr{RRLF}}) \\
				\svar{Sd}{t^+} &= t \end{aligned}$} \\
	\hline
	TD & $\obs{rf}{\attr{CURS}}{t} (\attr{PTD}$ + \sgn abs( \par
				 $\qquad \cldsv{\attr{CTST}_{Underlying}^{Contract}}{t}{Ipac}{\attr{CNTRL}=RPA} - $ \par
				 $\qquad \cldsv{\attr{CTST}_{Underlying}^{Contract}}{t}{Ipac}{\attr{CNTRL}=RPA,$ \par
					$\qquad\qquad\attr{RRLC}=\attr{RRLC},\attr{RRLF}=\attr{RRLF}}))$
		& {$\begin{aligned}
				\svar{Ipac}{t^+} &= 0.0 \\
				\svar{Sd}{t^+} &= t \end{aligned}$} \\
	\hline
	$z_\tau^m$ & $\sgn abs(\fev{x_\tau^m}-\fev{y_\tau^m})$ \par
			where $abs(u)$ defines that the absolute value of $u$ is taken.
		& {$\begin{aligned}
				\svar{Ipac}{t^+} &= 0.0 \\
				\svar{Sd}{t^+} &= t \end{aligned}$} \\
	\hline
	CE & \pof{CE}{PAM} & \stf{AD}{CAPFL} \\
\end{functions}


%%%%%%%%%%%%%%%%%%%% subsection: optns %%%%%%%%%%%%%%%%%%%%

\subsection{OPTNS: Option}\label{sec:optns}


% ---------------------- table: optns schedule ----------------------
\begin{schedule}{OPTNS}
	AD & & Same as PAM \\
	\hline
	PRD & & Same as PAM \\
	\hline
	TD & & Same as PAM \\
	\hline
	XD & $t^{XD} = \attr{MD}$ &  \\
	\hline
	STD & $t^{STD} = \begin{cases} \attr{MD} & \text{if}\quad \attr{STD}=\undef \\
					\attr{STD} & \text{else} \end{cases}$ &  \\
	\hline
	CE & & Same as PAM \\
\end{schedule}


% ---------------------- table: optns states ----------------------
\begin{states}{OPTNS}
	$\svar{Pos}{}$ & $\svar{Pos}{t_0} = \begin{cases}
							0.0 & \text{if}\quad t_0\leq\attr{MD} \\
							\max(S_{t_0}-\attr{OPS1},0) & \text{else if}\quad \attr{OPTP}=\text{'C'} \\
							\max(\attr{OPS1}-S_{t_0},0) & \text{else if}\quad \attr{OPTP}=\text{'P'} \\
							\max(S_{t_0}-\attr{OPS1},0) & \text{else} \\ \qquad + \max(\attr{OPS2}-S_{t_0},0) &  \end{cases}$ & with\par
			$S_{t_0}=\obs{rf}{\cldca{\attr{CTST}_{Underlying}^{Contract}}{MOC}}{t_0}$ \\
	\hline
	$\svar{Prf}{}$ & & Same as PAM \\
	\hline
	$\svar{Sd}{}$ & & Same as PAM \\
\end{states}


% ---------------------- table: optns functions ----------------------
\begin{functions}{OPTNS}
	AD & \pof{AD}{PAM} & {$\begin{aligned}
					\svar{Pos}{t^+} &= \begin{cases}
								\max(S_t-\attr{OPS1},0) & \text{if}\quad \attr{OPTP}=\text{'C'} \\
								\max(\attr{OPS1}-S_t,0) & \text{else if}\quad \attr{OPTP}=\text{'P'} \\
								\max(S_t-\attr{OPS1},0) & \text{else} \\ \qquad + \max(\attr{OPS2}-S_t,0) &  \end{cases} \\
					\svar{Sd}{t^+} &= t \end{aligned}$}\par
with\par
			$S_t=\obs{rf}{\cldca{\attr{CTST}_{Underlying}^{Contract}}{MOC}}{t}$ \\
	\hline
	PRD & $\obs{rf}{\attr{CURS}}{t}(-1)\attr{PPRD}$ & \stf{PRD}{STK} \\
	\hline
	TD & $\obs{rf}{\attr{CURS}}{t}\attr{PTD}$ & \stf{TD}{STK} \\
	\hline
	XD & 0.0 & {$\begin{aligned}
					\svar{Pos}{t^+} &= \begin{cases}
								\max(S_t-\attr{OPS1},0) & \text{if}\quad \attr{OPTP}=\text{'C'} \\
								\max(\attr{OPS1}-S_t,0) & \text{else if}\quad \attr{OPTP}=\text{'P'} \\
								\max(S_t-\attr{OPS1},0) & \text{else} \\ \qquad + \max(\attr{OPS2}-S_t,0) &  \end{cases} \\
					\svar{Sd}{t^+} &= t \end{aligned}$}\par
with\par
			$S_t=\obs{rf}{\cldca{\attr{CTST}_{Underlying}^{Contract}}{MOC}}{t}$ \\
	\hline
	STD & $\obs{rf}{\attr{CURS}}{t}\sgn\svar{Pos}{t^-}$ & {$\begin{aligned}
					\svar{Pos}{t^+} &= 0.0 \\
					\svar{Sd}{t^+} &= t \end{aligned}$} \\
	\hline
	CE & \pof{CE}{PAM} & \stf{AD}{OPTNS} \\
\end{functions}



%%%%%%%%%%%%%%%%%%%% subsection: futur %%%%%%%%%%%%%%%%%%%%

\subsection{FUTUR: Future}\label{sec:futur}


% ---------------------- table: futur schedule ----------------------
\begin{schedule}{FUTUR}
	AD & & Same as PAM \\
	\hline
	PRD & & Same as PAM \\
	\hline
	TD & & Same as PAM \\
	\hline
	XD & & Same as OPTNS \\
	\hline
	STD & & Same as OPTNS \\
	\hline
	CE & & Same as PAM \\
\end{schedule}


% ---------------------- table: futur states ----------------------
\begin{states}{FUTUR}
	$\svar{Pos}{}$ & $\svar{Pos}{t_0} = \begin{cases} 0.0 & \text{if}\quad t_0\leq\attr{MD} \\
							S_{t_0}-\attr{PFUT} & \text{else} \end{cases}$ & with\par
			$S_{t_0}=\obs{rf}{\cldca{\attr{CTST}_{Underlying}^{Contract}}{MOC}}{t_0}$ \\
	\hline
	$\svar{Prf}{}$ & & Same as PAM \\
	\hline
	$\svar{Sd}{}$ & & Same as PAM \\
\end{states}


% ---------------------- table: futur functions ----------------------
\begin{functions}{FUTUR}
	AD & \pof{AD}{PAM} & {$\begin{aligned}
					\svar{Pos}{t^+} &= S_t-\attr{PFUT} \\
					\svar{Sd}{t^+} &= t \end{aligned}$}\par
with\par
			$S_t=\obs{rf}{\cldca{\attr{CTST}_{Underlying}^{Contract}}{MOC}}{t}$ \\
	\hline
	PRD & \pof{PRD}{OPTNS} & \stf{PRD}{STK} \\
	\hline
	TD & \pof{TD}{OPTNS} & \stf{TD}{STK} \\
	\hline
	XD & \pof{XD}{OPTNS} & {$\begin{aligned}
					\svar{Pos}{t^+} &= S_t-\attr{PFUT} \\
					\svar{Sd}{t^+} &= t \end{aligned}$}\par
with\par
			$S_t=\obs{rf}{\cldca{\attr{CTST}_{Underlying}^{Contract}}{MOC}}{t}$ \\
	\hline
	STD & \pof{STD}{OPTNS} & \stf{STD}{OPTNS} \\
	\hline
	CE & \pof{CE}{PAM} & \stf{AD}{FUTUR} \\
\end{functions}


%%%%%%%%%%%%%%%%%%%% subsection: ceg %%%%%%%%%%%%%%%%%%%%

\subsection{CEG: Credit Enhancement Guarantee}\label{sec:ceg}


% ---------------------- table: ceg schedule ----------------------
\begin{schedule}{CEG}
	AD & & Same as PAM \\
	\hline
	IED & & Same as PAM \\
	\hline
	PRD & & Same as PAM \\
	\hline
	TD & & Same as PAM \\
	\hline
	FP & $\vec{t}^{FP} = \begin{cases} \undef & \text{if} \quad \attr{FER}=\undef \lor \attr{FER}=0 \\
					\sdl{s}{\attr{FPCL}}{T} & \text{else} \end{cases}$
		& with\par $s = \begin{cases} \undef & \text{if} \quad \attr{FPANX}=\undef \land \attr{FPCL}=\undef\\
					   \attr{PRD}+\attr{FPCL} & \text{else if} \quad \attr{FPANX} = \undef \\
					   \attr{FPANX} & \text{else} \end{cases}$ \par
			$T = \begin{cases} \attr{MD} & \text{if}\quad t^{XD} = \undef \\
					t^{XD} & \text{else} \end{cases}$ \\
	\hline
	XD & $t^{XD} = \inf \tev{e^{CE}}$ & with\par
		$e^{CE} = e\in\obs{ev}{i,\attr{CE}}{t_0} | typ(e)==\attr{CET}\wedge$\par
				\quad\quad $ \tev{e} \leq \svar{Tmd}{t_0}$\par
		and\par
		$i = \attr{CTST}_{CoveredContract}^{Contract}(i)$  \\
	\hline
	STD & $t^{STD} = t^{XD}$  \\
	\hline
	MD & $t^{MD}=\begin{cases} \svar{Tmd}{t_0} & \text{if}\quad t^{XD}=\undef \\
				\undef & \text{else} \end{cases}$ & \\
	\hline
	CE & & Same as PAM \\
\end{schedule}


% ---------------------- table: ceg states ----------------------
\begin{states}{CEG}
	$\svar{Tmd}{}$ & $\svar{Tmd}{t_0}=\begin{cases} \attr{MD} & \text{if}\quad \attr{MD}\neq\undef \\
				\max(\tev{\{e_t^k\}}) & \text{else} \end{cases}$
				& with\par
					{$\begin{aligned}
						\{e_t^k\} &= \{e_t^{k,1}\}\cup\{e_t^{k,2}\}\cup ...\cup\{e_t^{k,n}\} \\
						\{e_t^{k,i}\} &= \cldev{\attr{CTST}_{CoveredContract}^{Contract}(i)}{t_0}\\
						n &= \mid \attr{CTST}_{CoveredContract}^{Contract} \mid \end{aligned}$} \\
	\hline
	$\svar{Nt}{}$ & $\svar{Nt}{t_0} = \begin{cases}
							0.0 & \text{if}\quad t_0\geq\svar{Tmd}{t_0} \\
							\attr{CECV}\sgn\attr{NT} & \text{else if}\quad \attr{NT}\neq\undef \\
							\attr{CECV}\sgn\sum_{i=1}^{\mid I\mid} n_i & \text{else} \end{cases}$ & with\par
			$n_i=\begin{cases} \cldsv{I(i)}{t_0}{Nt}{x} & \text{if}\quad \attr{CEGE}=NO \\
					\cldsv{I(i)}{t_0}{Nt}{x} & \text{else if}\quad \attr{CEGE}=NI \\
					\quad + \cldsv{I(i)}{t_0}{Ipac}{x} & \\
					\obs{rf}{\cldca{I(i)}{MOC}}{t_0} & \text{else} \end{cases}$ \par
	where\par
	$I=\attr{CTST}_{CoveredContract}^{Contract}$ \\
	\hline
	$\svar{Fac}{}$ & $\svar{Fac}{t_0} = \begin{cases} 0.0 & \text{if} \quad \attr{FER}=\undef \\
					\attr{FEAC} & \text{else if} \quad \attr{FEAC} \neq \undef \\
					\svar{Nt}{t_0}\yfr{t^-}{t_0}\attr{FER} & \text{else if} \quad \attr{FEB}=\text{'N'} \\
					\frac{\yfr{t^{FP-}}{t_0}}{\yfr{t^{FP-}}{t^{FP+}}}\sgn\attr{FER} & \text{else} \end{cases}$ &
			with\par
			{$\begin{aligned}
						t^{FP-} &= \sup t \in \vec{t}^{FP}\mid t<t_0 \\
						t^{FP+} &= \inf t \in \vec{t}^{FP}\mid t>t_0 \end{aligned}$} \\
  	\hline
	$\svar{Prf}{}$ & & Same as PAM \\
	\hline
	$\svar{Sd}{}$ & & Same as PAM \\
\end{states}


% ---------------------- table: ceg functions ----------------------
\begin{functions}{CEG}
	AD & \pof{AD}{PAM} & {$\begin{aligned}
					\svar{Nt}{t^+} &= \begin{cases}
							0.0 & \text{if}\quad t\geq\svar{Tmd}{t_0} \\
							\svar{Nt}{t^-} & \text{else if}\quad \attr{NT}\neq\undef \\
							\attr{CECV}\sum_{i=1}^{\mid I\mid} n_i & \text{else} \end{cases} \\
					\svar{Fac}{t^+} &= \begin{cases} \svar{Fac}{t^-} + \yfr{\svar{Sd}{t^-}}{t}\svar{Nt}{t^-}\attr{FER} & \text{if} \quad \attr{FEB}=\text{'N'} \\
					\frac{\yfr{t^{FP-}}{t}}{\yfr{t^{FP-}}{t^{FP+}}}\sgn\attr{FER} & \text{else} \end{cases} \\
					\svar{Sd}{t^+} &= t \end{aligned}$}\par
with\par
			{$\begin{aligned} n_i &= \begin{cases} \cldsv{I(i)}{t}{Nt}{} & \text{if}\quad \attr{CEGE}=NO \\
					\cldsv{I(i)}{t}{Nt}{} & \text{else if}\quad \attr{CEGE}=NI \\
					\quad + \cldsv{I(i)}{t}{Ipac}{} & \\
					\obs{rf}{\cldca{I(i)}{MOC}}{t} & \text{else} \end{cases} \\
					t^{FP-} &= \sup t \in \vec{t}^{FP}\mid t<t_0 \\
					t^{FP+} &= \inf t \in \vec{t}^{FP}\mid t>t_0 \end{aligned}$}\par
					where\par
	$I=\attr{CTST}_{CoveredContract}^{Contract}$ \\
	\hline
	PRD & \pof{PRD}{STK} & {$\begin{aligned}
					\svar{Nt}{t^+} &= \begin{cases}
							\svar{Nt}{t^-} & \text{if}\quad \attr{NT}\neq\undef \\
							\attr{CECV}\sum_{i=1}^{\mid I\mid} n_i & \text{else} \end{cases} \\
					\svar{Fac}{t^+} &= \sgn\attr{FEAC} \\
					\svar{Sd}{t^+} &= t \end{aligned}$}\par
with\par
			{$\begin{aligned} n_i &= \begin{cases} \cldsv{I(i)}{t}{Nt}{} & \text{if}\quad \attr{CEGE}=NO \\
					\cldsv{I(i)}{t}{Nt}{} & \text{else if}\quad \attr{CEGE}=NI \\
					\quad + \cldsv{I(i)}{t}{Ipac}{} & \\
					\obs{rf}{\cldca{I(i)}{MOC}}{t} & \text{else} \end{cases} \\
					t^{FP-} &= \sup t \in \vec{t}^{FP}\mid t<t_0 \\
					t^{FP+} &= \inf t \in \vec{t}^{FP}\mid t>t_0 \end{aligned}$}\par
					where\par
	$I=\attr{CTST}_{CoveredContract}^{Contract}$ \\
	\hline
	FP & {$\begin{aligned} &\obs{rf}{\attr{CURS}}{t}\sgn\attr{FER} & \text{if} \quad \attr{FEB}=A \\
				&\obs{rf}{\attr{CURS}}{t}\left(\svar{Fac}{t^-} + \svar{Nt}{t^-}\yfr{t^-}{t}\attr{FER}\right) & \text{else} \end{aligned}$}
			& {$\begin{aligned}
					\svar{Nt}{t^+} &= \begin{cases}
							\svar{Nt}{t^-} & \text{if}\quad \attr{NT}\neq\undef \\
							\attr{CECV}\sum_{i=1}^{\mid I\mid} n_i & \text{else} \end{cases} \\
					\svar{Fac}{t^+} &= 0.0 \\
					\svar{Sd}{t^+} &= t \end{aligned}$}\par
with\par
			{$\begin{aligned} n_i &= \begin{cases} \cldsv{I(i)}{t}{Nt}{} & \text{if}\quad \attr{CEGE}=NO \\
					\cldsv{I(i)}{t}{Nt}{} & \text{else if}\quad \attr{CEGE}=NI \\
					\quad + \cldsv{I(i)}{t}{Ipac}{} & \\
					\obs{rf}{\cldca{I(i)}{MOC}}{t} & \text{else} \end{cases} \\
					t^{FP-} &= \sup t \in \vec{t}^{FP}\mid t<t_0 \\
					t^{FP+} &= \inf t \in \vec{t}^{FP}\mid t>t_0 \end{aligned}$}\par
					where\par
	$I=\attr{CTST}_{CoveredContract}^{Contract}$ \\
	\hline
	XD & \pof{XD}{OPTNS} & {$\begin{aligned}
					\svar{Nt}{t^+} &= \begin{cases}
							\svar{Nt}{t^-} & \text{else if}\quad \attr{NT}\neq\undef \\
							\attr{CECV}\sum_{i=1}^{\mid I\mid} n_i & \text{else} \end{cases} \\
					\svar{Fac}{t^+} &= \begin{cases} \svar{Fac}{t^-} + \yfr{\svar{Sd}{t^-}}{t}\svar{Nt}{t^-}\attr{FER} & \text{if} \quad \attr{FEB}=\text{'N'} \\
					\frac{\yfr{t^{FP-}}{t}}{\yfr{t^{FP-}}{t^{FP+}}}\sgn\attr{FER} & \text{else} \end{cases} \\
					\svar{Sd}{t^+} &= t \end{aligned}$}\par
with\par
			{$\begin{aligned} n_i &= \begin{cases} \cldsv{I(i)}{t}{Nt}{} & \text{if}\quad \attr{CEGE}=NO \\
					\cldsv{I(i)}{t}{Nt}{} & \text{else if}\quad \attr{CEGE}=NI \\
					\quad + \cldsv{I(i)}{t}{Ipac}{} & \\
					\obs{rf}{\cldca{I(i)}{MOC}}{t} & \text{else} \end{cases} \\
					t^{FP-} &= \sup t \in \vec{t}^{FP}\mid t<t_0 \\
					t^{FP+} &= \inf t \in \vec{t}^{FP}\mid t>t_0 \end{aligned}$}\par
					where\par
	$I=\attr{CTST}_{CoveredContract}^{Contract}$ \\
	\hline
	STD & $\obs{rf}{\attr{CURS}}{t}\left(\svar{Nt}{t^-}+\svar{Feac}{t^-}\right)$ & {$\begin{aligned}
					\svar{Nt}{t^+} &= 0.0 \\
					\svar{Feac}{t^+} &= 0.0 \\
					\svar{Sd}{t^+} &= t \end{aligned}$} \\
	\hline
	MD & 0.0 & {$\begin{aligned}
				\svar{Nt}{t^+} &= 0.0 \\
				\svar{Sd}{t^+} &= t \end{aligned}$} \\
	\hline
	CE & \pof{CE}{PAM} & \stf{AD}{CEG} \\
\end{functions}



%%%%%%%%%%%%%%%%%%%% subsection: cec %%%%%%%%%%%%%%%%%%%%

\subsection{CEC: Credit Enhancement Collateral}\label{sec:cec}


% ---------------------- table: cec schedule ----------------------
\begin{schedule}{CEC}
	AD & & Same as PAM \\
	\hline
	XD & & Same as CEG \\
	\hline
	STD & & Same as CEG \\
	\hline
	MD & & Same as CEG \\
\end{schedule}


% ---------------------- table: cec states ----------------------
\begin{states}{CEC}
	$\svar{Tmd}{}$ & $\svar{Tmd}{t_0}=\begin{cases} \attr{MD} & \text{if}\quad \attr{MD}\neq\undef \\
				\max(\tev{\{e_t^k\}}) & \text{else} \end{cases}$
				& with\par
					{$\begin{aligned}
						\{e_t^k\} &= \{e_t^{k,1}\}\cup\{e_t^{k,2}\}\cup ...\cup\{e_t^{k,n}\} \\
						\{e_t^{k,i}\} &= \cldev{\attr{CTST}_{CoveredContract}^{Contract}(i)}{t_0} \\
						n &= \mid \attr{CTST}_{CoveredContract}^{Contract} \mid \end{aligned}$} \\
	\hline
	$\svar{Nt}{}$ & $\svar{Nt}{t_0} = \begin{cases}
							0.0 & \text{if}\quad t_0\geq\svar{Tmd}{t_0} \\
							\min\left(\sum_{j=1}^{\mid J\mid} v_j, \attr{CECV}\sum_{i=1}^{\mid I\mid} n_i\right) & \text{else} \end{cases}$ &  with\par
			{$\begin{aligned}
				n_i &= \begin{cases} \cldsv{I(i)}{t_0}{Nt}{} & \text{if}\quad \attr{CEGE}=NO \\
					\cldsv{I(i)}{t_0}{Nt}{} & \text{else if}\quad \attr{CEGE}=NI \\
					\quad + \cldsv{I(i)}{t_0}{Ipac}{} & \\
					\obs{rf}{\cldca{I(i)}{MOC}}{t_0} & \text{else} \end{cases} \\
				v_j &= \obs{rf}{\cldca{J(j)}{MOC}}{t_0} \\
				I &= \attr{CTST}_{CoveredContract}^{Contract} \\
				J &= \attr{CTST}_{CoveringContract}^{Contract}\end{aligned}$} \\
	\hline
	$\svar{Sd}{}$ & & Same as PAM \\
\end{states}


% ---------------------- table: cec functions ----------------------
\begin{functions}{CEC}
	AD & \pof{AD}{PAM} & {$\begin{aligned}
					\svar{Nt}{t^+} &= \begin{cases}
							0.0 & \text{if}\quad t\geq\svar{Tmd}{t_0} \\
							\min\left(\sum_{j=1}^{\mid J\mid} v_j, \attr{CECV}\sum_{i=1}^{\mid I\mid} n_i\right) & \text{else} \end{cases} \\
					\svar{Sd}{t^+} &= t \end{aligned}$}\par
 with\par
			{$\begin{aligned}
				n_i &= \begin{cases} \cldsv{I(i)}{t_0}{Nt}{} & \text{if}\quad \attr{CEGE}=NO \\
					\cldsv{I(i)}{t_0}{Nt}{} & \text{else if}\quad \attr{CEGE}=NI \\
					\quad + \cldsv{I(i)}{t_0}{Ipac}{} & \\
					\obs{rf}{\cldca{I(i)}{MOC}}{t_0} & \text{else} \end{cases} \\
				v_j &= \obs{rf}{\cldca{J(j)}{MOC}}{t_0} \\
				I &= \attr{CTST}_{CoveredContract}^{Contract} \\
				J &= \attr{CTST}_{CoveringContract}^{Contract}\end{aligned}$} \\
	\hline
	XD & \pof{XD}{OPTNS} & {$\begin{aligned}
					\svar{Nt}{t^+} &= \min\left(\sum_{j=1}^{\mid J\mid} v_j, \attr{CECV}\sum_{i=1}^{\mid I\mid} n_i\right) \\
					\svar{Sd}{t^+} &= t \end{aligned}$}\par
 with\par
			{$\begin{aligned}
				n_i &= \begin{cases} \cldsv{I(i)}{t_0}{Nt}{} & \text{if}\quad \attr{CEGE}=NO \\
					\cldsv{I(i)}{t_0}{Nt}{} & \text{else if}\quad \attr{CEGE}=NI \\
					\quad + \cldsv{I(i)}{t_0}{Ipac}{} & \\
					\obs{rf}{\cldca{I(i)}{MOC}}{t_0} & \text{else} \end{cases} \\
				v_j &= \obs{rf}{\cldca{J(j)}{MOC}}{t_0} \\
				I &= \attr{CTST}_{CoveredContract}^{Contract} \\
				J &= \attr{CTST}_{CoveringContract}^{Contract}\end{aligned}$} \\
	\hline
	STD & $\obs{rf}{\attr{CURS}}{t}\svar{Nt}{t^-}$ & {$\begin{aligned}
					\svar{Nt}{t^+} &= 0.0 \\
					\svar{Sd}{t^+} &= t \end{aligned}$} \\
	\hline
	MD & 0.0 & {$\begin{aligned}
				\svar{Nt}{t^+} &= 0.0 \\
				\svar{Sd}{t^+} &= t \end{aligned}$} \\
\end{functions}


%%%%%%%%%%%%%%%%%%%% section: bibliography %%%%%%%%%%%%%%%%%%%%

\bibliography{bibliography}


%%%%%%%%%%%%%%%%%%%% end doc %%%%%%%%%%%%%%%%%%%%

\end{document}
